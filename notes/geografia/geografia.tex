\documentclass[a4paper]{article}

\usepackage{amsmath}
\usepackage{amssymb}
\usepackage{parskip}
\usepackage{fullpage}
\usepackage{hyperref}
\usepackage{chronology}
\usepackage{stellar}
\usepackage{bettelini}
\usepackage{tikz}

\usetikzlibrary{cd}

\hypersetup{
    colorlinks=true,
    linkcolor=black,
    urlcolor=blue,
    pdftitle={geografia},
    pdfpagemode=FullScreen,
}

\title{geografia}
\author{Paolo Bettelini}
\date{}

\begin{document}

\maketitle
\tableofcontents
\pagebreak

\part{Geografia Economica}

\section{Esercizio: la Francia ferita nell'epoca delle policrisi, Edgar Morin}

\textbf{Identifica e sintetizza in una linea del tempo i principali riferimemti storici}

\begin{chronology}[25]{1789}{2023}{\textwidth}
    \event{1789}{Rivoluzione Francese (1789)}
    \event{1830}{Rivoluzioni e crisi (1830-1906)}
    \event{1901}{Multiculturalismo francese (1901)}
    \event{1914}{Prima guerra mondiale (1901)}
    \event{1929}{Crisi economia (1929)}
    \event{1939}{Seconda guerra mondiale (1939)}
    \event{1947}{Guerra Fredda (1947)}
    \event{1956}{Crisi del comunismo (1956)}
    \event{1968}{Rivolta studentesca (1968)}
    \event{1974}{National Rally (1972)} % cambiato anno per non sovrascrivere
    \event{1989}{Caduta Muro Berlino (1989)}
    \event{2014}{Guerra Russo-Ucrania (2014)}
\end{chronology}

\textbf{Identifica i riferimenti spaziali e specifica quali sono le scale geografiche mobilitate dall'autore}

L'autore cita riferimenti spaziali su tre diverse scale geografiche.
Vengono citate per la scala nazionale nazioni con grandi influenze politiche, quali la Francia,
Russia e Stati Uniti.
Vengono citate l'Europa, Africa o l'Occidente.
Come scala più ampia viene citata quella globale.

\textbf{Definisci brevemente il termine "policrisi" spiegando l'evoluzione del concetto}

\sdefinition{Policrisi}{
    Una \textit{policrisi} è un insieme di molteplici event nefasti e interdipendenti che potrebbero
    portare a danni su grande scala (planetaria).
}

La "policrisi" sottolinea l'idea che il mondo moderno sia
caratterizzato da una profonda interconnettività.

\textbf{Proponi una riflessione argomentata sui contenuti delle ultime 4 righe del testo}

Evitare che la Francia (la Repubblica) si trasformi in uno stato del controllo
è uno degli step fondamentali per ridurre la policrisi,
essendo la Francia un importante tassello della policrisi globale.

\textbf{Riassumi in due (o tre) frasi il contenuto del testo}

Le relazioni fra eventi di scale diverse, sia a livello locale che a quello globale, 
sono interdipendenti e portano a situazioni di crisi globali.

Eventi che sono all'apparenza locali possono avere grandi effetti in crisi di scala maggiore.

\pagebreak

\section{Secolo}

\sdefinition{Secolo}{
    Un \textit{secolo} può essere definito in mainera stretta come 100 anni,
    oppure come un periodo di circa 100 anni con caratteristiche omogenee.
}

% https://moodle.edu.ti.ch/libe/pluginfile.php/114669/mod_resource/content/0/Presentazione%20Secolo%20breve.pdf

% https://moodle.edu.ti.ch/libe/pluginfile.php/114664/mod_folder/content/0/1Viola%20-%20La%20crisi%20del%2029.pdf?forcedownload=1

\pagebreak

\part{Geografia Fisica}

\section{L'interazione tra le sfere terrestri}

La terra può essere suddivisa in diverse sfere interdipendenti.

\sdefinition{Atmosfera}{
    componente gassosa che avvolge il pianeta. L'aria è inodore e incolore ed è quasi 800 volte meno densa dell'acqua.
}

\sdefinition{Idrosfera}{
    insieme di tutte le acque del pianeta nei diversi stati di aggregazione; comprende le acque marine e quelle continentali.
}
    
\sdefinition{Litosfera}{
    componente solida superficiale costituita dalle rocce.
}

\sdefinition{Biosfera}{
    componente vivente e comprende gli organismi che popolano la zona di interazione delle tre sfere precedenti.
}

% https://tikzcd.yichuanshen.de/#N4Igdg9gJgpgziAXAbVABwnAlgFyxMJZARgBpiBdUkANwEMAbAVxiRAB12cYAPHYAEL44AMxgAnOgF8QU0uky58hFGQAMVWoxZtO3PsACCOALaYxkmXIXY8BIgCZSDzfWatEHLr34AZXOYS0rLyIBi2ykRqzq7aHl76-ACSUOKBlrKaMFAA5vBEoCJpJkhkIDgQSE4gAEYwYFBIAMxq1iBFECWIZRXN1HUNSAC0LW0dXU3UvYjVA42II62h40jR5ZXd-fXzo8vFpVMba3PNS4X7M4d9tdvDu+ed19PHtwv37Rdr05M3g29nH0eiC+G1mr3eK0u61WWz+EIu1WmZROCwALABOMYXH5I2HzDHUBhYMDxOAQImNagACxgdHmYCYDAYUzoWAYbEgJMyUiAA
%\begin{tikzcd}
%    & \text{Atmosfera} \arrow[rdd, bend left] \arrow[ldd, bend right] \arrow[d, bend left] &                                                                                          \\
%    & \text{Biosfera} \arrow[u, bend left] \arrow[ld, bend right] \arrow[rd, bend left]    &                                                                                          \\
%\text{Idrosfera} \arrow[rr, bend right] \arrow[ru, bend right] \arrow[ruu, bend left=49] &                                                                                      & \text{Litosfera} \arrow[ll, bend right] \arrow[lu, bend left] \arrow[luu, bend right=49]
%\end{tikzcd}

I processi che coinvologno le sfere terrestri sono di due tipi:

\sdefinition{Processo Esogeno}{
    Sono processi attivati dall'energia del Sole e avvengono sulla superficie terrestre; sono per esempio i movimento delle acquem, i passaggi di stato, il tempo atmosferi o il modellamento della superficie terrestre.
}

\sdefinition{Processo Endogeno}{
    Sono processi attivati dal calore interno della Terra; sono i processi chde portano alla formazione di catene montuose e di nuovi oceani.
}

\end{document}
