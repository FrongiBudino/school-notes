\documentclass[a4paper]{article}

\usepackage{amsmath}
\usepackage{amssymb}
\usepackage{parskip}
\usepackage{fullpage}
\usepackage{hyperref}
\usepackage{chronology}
\usepackage{stellar}

\hypersetup{
    colorlinks=true,
    linkcolor=black,
    urlcolor=blue,
    pdftitle={geografia},
    pdfpagemode=FullScreen,
}

\title{geografia}
\author{Paolo Bettelini}
\date{}

\begin{document}

\maketitle
\tableofcontents
\pagebreak

\part{Geografia Economica}

\section{Esercizio: la Francia ferita nell'epoca delle policrisi, Edgar Morin}

\textbf{Identifica e sintetizza in una linea del tempo i principali riferimemti storici}

\begin{chronology}[25]{1789}{2023}{\textwidth}
    \event{1789}{Rivoluzione Francese (1789)}
    \event{1830}{Rivoluzioni e crisi (1830-1906)}
    \event{1901}{Multiculturalismo francese (1901)}
    \event{1914}{Prima guerra mondiale (1901)}
    \event{1929}{Crisi economia (1929)}
    \event{1939}{Seconda guerra mondiale (1939)}
    \event{1947}{Guerra Fredda (1947)}
    \event{1956}{Crisi del comunismo (1956)}
    \event{1968}{Rivolta studentesca (1968)}
    \event{1974}{National Rally (1972)} % cambiato anno per non sovrascrivere
    \event{1989}{Caduta Muro Berlino (1989)}
    \event{2014}{Guerra Russo-Ucrania (2014)}
\end{chronology}

\textbf{Identifica i riferimenti spaziali e specifica quali sono le scale geografiche mobilitate dall'autore}

L'autore cita principalmente diverse nazioni con grandi influenze politiche, quali la Francia,
Russia e Stati Uniti.
Le scale geografiche citate sono quindi principalmente nazionali,
ma le conseguenze degli eventi citati si espandono a livello globale.

\textbf{Definisci brevemente il termine "policrisi" spiegando l'evoluzione del concetto}

\sdefinition{Policrisi}{
    Una \textit{policrisi} è un insieme di molteplici event nefasti e interdipendenti che potrebbero
    portare a danni su grande scala.
}

La "policrisi" sottolinea l'idea che il mondo moderno sia
caratterizzato da una profonda interconnettività.

L'evoluzione del concetto...

\textbf{Proponi una riflessione argomentata sui contenuti delle ultime 4 righe del testo}

Evitare che la Francia (la Repubblica) si trasformi in uno stato del controllo
è uno degli step fondamentali per ridurre la policrisi,
essendo questo uno dei problemi principali.

\textbf{Riassumi in due (o tre) frasi il contenuto del testo}

Le relazioni fra eventi di scale diverse, sia a livello locale che a quello globale, 
sono interdipendenti e portano a situazioni di crisi globali.

\pagebreak

\part{Geografia Fisica}

\end{document}
