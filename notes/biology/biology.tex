\documentclass[a4paper]{article}

\usepackage{amsmath}
\usepackage{amssymb}
\usepackage{parskip}
\usepackage{fullpage}
\usepackage{hyperref}
\usepackage{stellar}

\hypersetup{
    colorlinks=true,
    linkcolor=black,
    urlcolor=blue,
    pdftitle={Biology},
    pdfpagemode=FullScreen,
}

\newcommand{\quotes}[1]{``#1''}

\newcommand\hr{\par\vspace{-.5\ht\strutbox}\noindent\hrulefill\par\vspace{0.15cm}}

\title{Biology}
\author{Paolo Bettelini}
\date{}

\begin{document}

\maketitle
\tableofcontents
\pagebreak

% 978 88 6364 9437 (trovato)
% 978 88 6364 9635
% 978 88 6364 9659

\section{Sistemi}

\sdefinition{Sistema}{
    Un \textit{sistema} (vivente e non-vivente) è composto di parti differenti, specializzate e interdipendenti. 
    
    \begin{enumerate}
        \item Organizzazione della relazione fra le parti
        \item Struttura fisica, chimica etc. 
        \item Processo di riproduzione
    \end{enumerate}
}

\sdefinition{Emergenza Sistemica}{
    Una \textit{emergenza sistemica} è lo scopo che le diverse parti riescono ad raggiungere ed eseguire.
}

\sdefinition{Molecola organica}{
    Una molecola organica contiene il carbonio (tranne \(CO_2\)).
    % più o meno
}

\subsection{Sistemi viventi}

\sdefinition{ATP}{
    ATP è un composto organico che provvede energia alle cellule per le loro funzioni.
}

I seguenti processi sono eseguiti da tutti gli organismi viventi.

\textbf{Nutrizione:}
Tutti gli organismi viventi si nutrono con del \quotes{cibo}, ossia materia.
In generale, gli esseri viventi necessitano di \(C\), \(O\), \(H\), \(N\), \(S\) e \(P\).
L'unico nutrimento della pianta è \(CO_2\) (materia inorganica), mentre
i nutrimenti degli animali sono materia organica.

\sdefinition{Autotrofo}{
    Un organismo \textit{autotrofo} può svolgere la propria funzione di nutrizione,
    elaborando alimenti inorganici mediante assunzione d'energia dal mondo inorganico.
}

\sdefinition{Eterotrofo}{
    Un organismo \textit{eterotrofo}
    si nutre di sostanze organiche prodotte dagli organismi autotrofi.
}

\textbf{Respirazione:}

Tutti gli organismi viventi respirano
\[
    C_6H_{12}O_6 + O_2 \rightarrow CO_2 + H_2O
\]
In assenza di ossigeno (si usa la materia organica per produrre energia), e alcuni organismi \textit{fermentano}.
Nel caso degli umani i muscoli respirano, se non c'è \(O\) fermentano e producono acido lattico
che deve successivamente essere smaltito.

Le piante respirano mediante la fotosintesi
\[
    CO_2 + H_2O \rightarrow C_6H_{12}O_6 + O_2
\]

\textbf{Si riproduce e ha un ciclo vitale}

\textbf{Evolve}

\textbf{È sensibile (sa rispondere all'ambiente)}

\textbf{Mantiene stabili le sue condizioni interne}

\hr

\sdefinition{Biotico}{
    Con \textit{biotico} si intende tutto ciò che è vivente o era vivente.
} % foglia morta, elefante

\sdefinition{Abiotico}{
    Con \textit{abiotico} si intende tutto ciò che non è vivente e non lo è mai stato.
} % roccia

\sdefinition{Detrito}{
    Con \textit{detrito} si intende il resto di ogni organismo vivente che è morto.
} % foglia morta

Il sistema vivente presenta le medesima ma caratteristiche del sistema non-vivente,
ma possiede anche le seguenti componenti.

\sdefinition{Componente}{
    Insieme di materia, concreta e tangibile
}

\sexample{Components}{
    Acqua, suolo, sali minerali, ossigeno.
}

\sdefinition{Fattore}{
    Derive dalla presenza di componenti, produce un determinato effetto o risultato e si può misurare.
}

\sexample{Fattore}{
    \begin{itemize}
        \item Decomposizione (fattore biotico).
        \item Predazione, catena alimentare (fattore biotico).
        \item Vento (fattore abiotico).
        \item Luce solare (fattore abiotico).
        \item Luce della lucciola (fattore biotico).
    \end{itemize}
}

Un fattore rappresenta tutto ciò che si può misurare e che non è una componente.

\subsubsection{Autopoiesi}

\sdefinition{Autopoiesi}{
    La capacità di ripararsi, modificarsi e riprodursi da solo, internamente ed in maniera autonoma.
}
I sistemi viventi sono organizzativamente chiusi, per cui hanno un confine.

\sexample{Sistema autopoietico - ciclo}{
    TODO: mettere foto
}

\sexample{Sistema autopoietico - cellula}{
    TODO: mettere foto
}

\subsubsection{Dissipazione}

\sdefinition{Dissipazione}{
    La necessità di consumare energia, materia ed informazioni dall'esterno.
}
I sistemi viventi sono metabolicamente aperti, per cui hanno degli scambi con l'esterno
e rinnovano il proprio materiale.

\subsubsection{Cognizione}

\sdefinition{Cognizione}{
    L'attiva conoscenza dell'ambiente, esterno ed interno, da parte del sistema.
}

\section{Biomolecole}

\sdefinition{Biomolecola}{
    Le \textit{biomolecole} sono le molecole dei processi biologici degli essere viventi.
}

Tutte le biomolecole contengono \(C\), \(O\) e \(H\).
Ci sono delle eccezioni, per esempio, gli idrocarburi contengono solamente \(C\) e \(O\).

Le biomolecole sono di 4 tipi:
\begin{itemize}
    \item Lipidi (grasso)
    \item Acidi nucleici (DNA e RNA)
    \item Carboidrati
    \item Proteine
\end{itemize}

Le macromolecole sono composte da \textit{monomeri} e \textit{polimeri}.
Nel corpo umano i polimeri sono creati dalle cellule mediante alle istruzioni nel DNA.
Le biomolecole fanno dei polimeri.

\sdefinition{Isomero}{
    Gli \textit{isomeri} sono delle molecole distinte con il medesimo numero di atomi,
    ma con una struttura diversa. Diversi isomeri potrebbero avere proprietà diverse.
}

\paragraph{Costruzione di polimeri}

Tutti i monomeri posseggono, da una parte un gruppo di idrogeno \(H\),
e dall'altra un gruppo \(OH\).
Due monomeri si uniscono mediante una reazione chimica chiamata \textit{condensazioe} o \textit{disidratazione}, la quale consiste
nell'unire un'estremità \(H\) con una \(OH\) mediante un legame.
La condensazione libera una molecola d'acqua come scarto.

\paragraph{Disintegrazione di polimeri}

Per separare un legame fra due monomeri, viene utilizzata la reazione chimica di \textit{idrolisi} o \textit{idratazione}.
Questa reazione necessita di una molecola di \(H_2O\).

\pagebreak

\subsection{Carboidrati}

\sdefinition{Carboidrato}{
    I \textit{carboidrati} sono dei tipi di biomolecole composti da carbonio, idrogeno e ossigeno
    \((CH_2O)_n\).
}

I monomeri di carboidrati si chiamano monosaccaridi.
I polimeri di carboidrati si chiamano polisaccaridi (disaccaridi, trisaccaridi)

\sexample{Maltosio}{
    Il maltosio sono 2 molecole di glucosio. Per unire 2 molecole di glucosio
    è necessario perderne una di \(H_2O\). Per cui il maltosio è dato da \(C_{12}H_{22}O_{11}\).
}

\sdefinition{Saccarosio}{
    Il \textit{saccarosio} è composto da un glucosio e un fruttosio (\(C_{12}H_{22}O_{11}\)).
}
\sdefinition{Lattosio}{
    Il \textit{lattosio} è composto da un glucosio e un galattosio (\(C_{12}H_{22}O_{11}\)).
}

I monosaccarisi sono glucosio, fruttosio, galattosio (isomeri).

%\section{Che cos'è la vita}
%\subsection{Visione sistemica}
%\subsection{Visione meccanicistica}

\end{document}