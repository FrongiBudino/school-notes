\documentclass[a4paper]{article}

\usepackage{amsmath}
\usepackage{amssymb}
\usepackage{parskip}
\usepackage{fullpage}
\usepackage{hyperref}

\hypersetup{
    colorlinks=true,
    linkcolor=black,
    urlcolor=blue,
    pdftitle={Fisica},
    pdfpagemode=FullScreen,
}

\title{Fisica}
\author{Paolo Bettelini}
\date{}

\begin{document}

\maketitle
\tableofcontents

\section{Grandezze fisiche}

\begin{center}
    \bgroup{}
    \def\arraystretch{1.25}
    \begin{tabular}{ |c|c|c|c| }
        \hline
        \textbf{Grandezza fisica} & \textbf{Definizioe} & \textbf{Misurazione} & \textbf{Unità} \\
        \hline
        \textbf{Massa} & Quantità di materia & Con una bilancia & \(Kg\) \\
        \hline
        \textbf{Peso} & Espressione di massa soggetta ad accelerazione & Con un dinamometro & \(N\) \\
        \hline
        \textbf{Volume} & Spazio tridimensionale occupato da una massa & XX & \(\frac{m^3}{Kg}\) \\
        \hline
    \end{tabular}
    \egroup{}
\end{center}

\end{document}
