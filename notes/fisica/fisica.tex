\documentclass[a4paper]{article}

\usepackage{amsmath}
\usepackage{amssymb}
\usepackage{parskip}
\usepackage{fullpage}
\usepackage{hyperref}
\usepackage{stellar}

\hypersetup{
    colorlinks=true,
    linkcolor=black,
    urlcolor=blue,
    pdftitle={Fisica},
    pdfpagemode=FullScreen,
}

\title{Fisica}
\author{Paolo Bettelini}
\date{}

\begin{document}

\maketitle
\tableofcontents

\pagebreak

\section{Forze}

\sdefinition{Costante di Coulomb}{
    La \textit{costante di Coulomb} è data da
    \[
        k = 9 \cdot 10^9 \frac{N \cdot m^2}{C^2} 
    \]
    dove \(C\) è l'unità di misura della carica elettrictà.
}

\sdefinition{Forza di Coulomb}{
    La \textit{forza di Coulomb} è la forze con la quale due cariche elettriche ferme,
    \(q_1\) e \(q_2\), a distanza \(r\), si attraggono
    \[
        F_Q = k \frac{q_1 q_2}{r^2}
    \]
    dove \(k\) è la costante di Coulomb.
}

\pagebreak

\section{Molle}

Due molle in parallelo hanno il medesimo allungamento,
mentre due molle in serie hanno la stessa forza.

% Spinta di Archimede
% Volume immerso = V liquido spostato
% Massa liquido spostato = Massa totale oggetto

\end{document}
