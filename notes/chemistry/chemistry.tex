\documentclass[a4paper]{article}

\usepackage{amsmath}
\usepackage{amssymb}
\usepackage{parskip}
\usepackage{fullpage}
\usepackage{hyperref}
\usepackage[backend=bibtex]{biblatex}

\hypersetup{
    colorlinks=true,
    linkcolor=black,
    urlcolor=blue,
    pdftitle={Chemistry},
    pdfpagemode=FullScreen,
}

\addbibresource{./references.bib}

\title{Chemistry}
\author{Paolo Bettelini}
\date{}

\begin{document}

\maketitle
\tableofcontents
\pagebreak

% 978.88.08.72527.1
% CHIMICA.BLU J.Brady

\section{Isotopi dell'idrogeno}

Il primo isotopo dell'idrogeno è il \textit{deuterio}, indicato con \(D\) o \(^2H\).
A differenza dell'idrogeno comune, il deuterio possiede un neutrone nel nucleo oltre al protone.
A causa di questa caratteristica, il deuterio ha una massa atomica leggermente superiore rispetto all'idrogeno normale.
Il deuterio è utilizzato in varie applicazioni, come nei reattori nucleari per la produzione di energia e come tracciatore in studi scientifici e biologici. 

Il secondo isotopo dell'idrogeno è è il trizio, indicato con \(T\) o \(^3H\).
A differenza dell'idrogeno comune, il deuterio possiede due neutroni nel nucleo oltre al protone.
A causa di questa composizione nucleare, il trizio ha una massa atomica maggiore rispetto agli altri isotopi dell'idrogeno.
Il trizio è radioattivo e decade nel tempo con una emivita di circa 12,3 anni, emettendo particelle beta.

\subsection{Acqua con deuterio e trizio}

È possibile ottenere dell'acqua, \(H_2O\), utilizzando gli isotopi \(D\) e \(T\) al posto di \(H\).

Queste sostanze sono chiamate \textit{acqua pesante} (\(D_2O\)) e
\textit{acqua superpesante} (\(T_2O\)).

\begin{center}
    \bgroup{}
    \def\arraystretch{1.25}
    \begin{tabular}{ |c|c|c|c| }
        \hline
        & \textbf{Acqua} & \textbf{Acqua pesante} & \textbf{Acqua Superpesante} \\
        \hline
        \textbf{Liquido (g/cm3)} & 0.997 & 1.11 & 1.20 \\
        \hline
        \textbf{Solido (g/cm3)} & 0.9168 & 1.105 & ? \\
        \hline
    \end{tabular}
    \egroup{}
\end{center}

Possiamo quindi notare come la versione solida dell'acqua pesante galleggi
nell'acqua normale \cite{deuterated-water}.

\nocite{*} % cite all entries

\printbibliography

\end{document}
