\documentclass[a4paper]{article}

\usepackage{amsmath}
\usepackage{amssymb}
\usepackage{parskip}
\usepackage{fullpage}
\usepackage{hyperref}
\usepackage{stellar}
\usepackage{soul}
\usepackage{tikz}
\usepackage{graphicx}

\usetikzlibrary{cd}

\hypersetup{
    colorlinks=true,
    linkcolor=black,
    urlcolor=blue,
    pdftitle={Italiano},
    pdfpagemode=FullScreen,
}

\title{Italiano}
\author{Paolo Bettelini}
\date{}

\newcommand{\quotes}[1]{``#1''}

\newcommand\hr{\par\vspace{-.5\ht\strutbox}\noindent\hrulefill\par\vspace{0.15cm}}

\begin{document}

\maketitle
\tableofcontents
\pagebreak

% Autori:
% Dante, Boccaccio, Petrarca
% Machiavelli (sagiistica politica)
% (sagiistica politica) e Ariosto
% Beccaria
% Leopardi
% Leopardi

% Esame scritto: 2 traccie nuove e sceglierne una
% commentare 4 ore di orologio

% Porta un dizionario
% Sufficientemente grande da avere le parole alma guiderdone spirto

% Verifiche:
% 11 ott, Dante 
% 6 dic, Boccaccio(?)

% TODO, spostare definizioni in un punto apposta



\part{Dante (1265-1321)}

\section{Biografia}

La biografia di Dante è molto offuscata e nessuno scritto originale è rimasto.
Questi fattori rendono difficile datare le sue varie opere e contestualizzarle. Inoltre,
è anche difficile validare in maniera precisa le parole esatte scritte dall'autore, siccome i testi che possediamo
sono frutto di trascrizioni.

Dante nasce a Firenze nel 1265 in una piccola nobilità cittadina.
A 12 anni diventa promesso sposo di a Gemma Donati.

A 18 anni incontra Beatrice, dopo averla vista per la prima volta a 9 anni.

Verso l'anno 1295 Dante, si avvicina alla politica.
Si iscrive all'Arte (Arte dei medici e degli speziali), questo è dato dal fatto che essere iscritti ad un Arte
fosse un requisito necessario per esercitare un'attività politica. Dante diventa piore, per cui a capo della cittadina. %%

Prima della nascita di Dante, i Ghibellini sostenevano il potere dell'imperatore, mentre i Guelfi sostenevano quello papa.
Le due parti erano in forte conflitto, e nella battaglia del 1266, muore il figlio dell'imperatore.
I Ghibellini escono quindi di scena quando Dante è appena nato.

Successivamente, i Guelfi si separano in Bianchi e Neri, con un conflitto ancora più forte di quello precedente.

La scena politica fiorentina era dominata dallo scontro fra i Bianchi e i Neri.
Dante, durante il suo priorato, manda in esilio in più violenti dei Neri, fino alla scaduta del suo priorato.
Papa Bonificio VIII manda le truppe di Carlo di Valois, le quali permettono ai Neri di prendere carica al governo.
I bianchi vengono quindi esiliati, fra cui Dante. 

% Boccaccio - primo grande studiso di dante

\pagebreak

\section{Vita Nova}

\subsection{Introduzione}

\sdefinition{Sonetto}{
    Il \textit{sonetto} è una poesia composto da 2 quartine e 2 terzine dove tutti i versi sono degli endecasillabi.
}

\sdefinition{Prosimetro}{
    Il \textit{prosimetro} è un testo ibrido, composto da un
    racconto (prosa) intervallato e poesie (versi).
}

La \textit{Vita nova} è il primo prosimetro. Esso racconta la storia d'amore da parte di Dante
nei confronti di Beatrice.
Questa vicenda diventa un modello per questa tipologia di narrativa.

Il significato del titolo indica come Dante consideri l'inizio della sua vita (nuova vita, rinnovata)
quando vide Beatrice per la prima volta.
Il primo contatto amoroso nella poesie è spesso caratterizzato da un innamoramento a prima vista.

% Il saluto nel medioevo da parte delle donne
Quando la voce dell'interesse di Dante nei confronti di Beatrice le giunge, lei gli nega il saluto.
\snote{Il saluto nel medioevo}{
    Il saluto nel medioevo ha un significato molto più profondo di quello odierno.
    %Infatti, 
}
Nonostante il rifiuto, Dante continua ad esprimere il suo amore verso Beatrice semplicemente
lodandola (scrivendo di lei), completamente senza ricambio di interesse.
Questa loda rappresenta la forma più pura di amore.

Questo libro introduce la simbologia del numero 9 associato a Beatrice.
Ciò è dato dal fatto che Dante l'abbia vista per la prima volta a 9 anni, rivista 9 anni dopo,
e altri motivi che vengono descritti. Il numero 9 è anche un simbolo biblico (3 volte la trinità).

\subsection{Oltre la spera}

Il seguente sonetto descrive il concetto di \textbf{intelligenza nova}
indotta nello spirito di Dante.

\begin{center}
    \textit{Oltre la spera che più larga gira,} \\
    \textit{passa 'l sospiro ch'esce del mio core:} \\
    \textit{intelligenza nova, che l'Amore} \\
    \textit{piangendo mette in lui, pur sù lo tira.}
\end{center}

Il primo verso è una perifrasi che indica \quotes{oltre il pianeta più lontano} (chiamato \textit{Il Primo Nobile}), ossia il paradiso
siccome la visione dell'universo era quella tolemaica e creazionista.

Il secondo verso ci indica che il sospiro del poeta esce dal suo cuore, mentre è vivo, dalla sua intimità più profonda,
e attraverso i cieli fino al paradiso.  

Ai versi 3-4 viene descritto ciò che permette questo percorso, ossia ciò che lo tira
verso l'alto. Questa forza è un'intelligenza nova, ossia una nuova sensibilità nel vedere le cose.
Questa nuova intelligenza deriva dall'amore, che permette all'autore di avere una nuova consapevolezza.
Questa esperienza amorosa è dolorosa ma porta ad una nuova capacità di intendimento.
Inoltre, la parola amore ha la maiuscola perché esso viene personificato.

\begin{center}
    \textit{Quand'elli è giunto là dove disira,} \\
    \textit{\textbf{vede} una donna, che riceve onore,} \\
    \textit{e \textbf{luce} sì, che per lo suo \textbf{splendore}} \\
    \textit{lo peregrino spirito la \textbf{mira}.}
\end{center}

La seconda quartina descrive il punto di arrivo.
Quando lo spirito arriva, vede una donna, la quale viene onorata dagli altri beati, Dio e la Madonna.
Viene anche detto che questa donna brilla.
A causa di questo grande splendore, lo spirito giunto in paradiso (pellegrino, in pellegrinaggio) la ammira.

\begin{center}
    \textit{\textbf{Vedela} tal, che quando 'l mi ridice,} \\
    \textit{io no lo \textbf{'ntendo}, sì \textbf{parla} sottile} \\
    \textit{al cor dolente, che lo fa \textbf{parlare}.}
\end{center}

Lo spirito ripercorre il medesimo tragitto verticale, ma al contrario, tornando da Dante.
Questo spirito cerca di spiegargli che cosa ha visto.
\quotes{La vede tale che quando me lo ridice, io non capisco}.
Dante non comprende quindi ciò che lo spirito gli riferisce, perché
\quotes{parla sottile}, ossia parla in maniera troppo difficile.
Il cuore dolente del poeta è ciò lo fa sì che lo spirito venga interrogato.
Infatti, lo spirito parla proprio al cuore \underline{e} a Dante (questo amplifica l'incomprensione della spiegazione).
Lo spirito parla in maniera troppo complessa perché il linguaggio non riesce
ad esprimere quello che si è provato (topos dell'ineffabilità, è ineffabile)
siccome l'esperienza lo tracende.

\begin{center}
    \textit{So io che \textbf{parla} di quella gentile,} \\
    \textit{però che spesso \textbf{ricorda} Beatrice,} \\
    \textit{\st{sì ch'io lo \textbf{'ntendo} ben, donne mie care.}}
\end{center}

\textbf{\color{red}Nota:} La parola \quotes{però} significa \quotes{per ciò}. \\
Questa è l'unica occorrenza dove Beatrice viene nominata direttamente in un testo poetico in \textit{Vita Nova}.\\
Nell'incomprensione fra Dante e lo spirito, Dante capisce che la donna vista era sicuramente
Beatrice. \\
Nella poesia antica, la parola \textit{gentile} è molto più profonda di quella odierna
e possiede un significato diverso. Essa ha un significato nobile di purezza (nobiltà d'animo).

L'ultimo verso è dato dal fatto che Dante si stesse riferendo a delle Donne nel testo.

\hr

Questo sonetto è diviso in due parti, dove vengono distinte le due verticalità del viaggio dello spirito
(avanti e indietro).

Molte parole della prima parte appartendono alla sfera visiva, poiché il paradiso
è fatto di luci, mentre molte parole della seconda fanno parte del parlare.
Questo è dato dal fatto che lo spirito può vedere, ma ha l'impossibilità di esprimersi.

Questa separazione è collegata dall'uso di due parole quasi uguali,
\textbf{mira} e \textbf{Vedela} (detto per anadiplosi).

% TODO atoni e tonici

\sdefinition{Legge di Tobler Mussafia}{
    È vietato iniziare un verso (poesie o prosa) o far seguire una congiunzione coordinante
    con un pronome atoni.
}

\pagebreak

\section{La Divina Commedia}

\subsection{Cosmo Dantesco}

\sdefinition{Sistema tolemaico}{
    Data la credenza creazionista, Dio ha creato l'uomo e l'ha collocato al centro.
    Per cui, la Terra risiede al centro del sistema solare, dove gli altri pianeti gli ruotano attorno.
}

Le colonne d'Ercole (Stretto di Gibilterra) e La foce del Gange
sono i due estremi della Terra. All'uomo non è concesso conoscere oltre questi confini
(fare ciò implicherebbe peccare di superbia).

Gerusalemme si trova al centro dell'emisfero. Sotto di esso, risiede l'inferno.

Lucifero era l'angelo prediletto di Dio.
Lucifero si ribella a Dio, e per punizione viene scagliato sulla Terra, la quale,
prova ribrezzo e si ritira formando la forma conica dell'inferno. Lucifero si trova nel punto
più profondo dell'inferno, ossia il centro della Terra, nonché il punto più lontano da Dio.

La creazione dell'inferno crea una montagna dall'altra parte del mondo, dove in cima ad esso
vi è il Giardino dell'Eden. Ciò marca anche la creazione del purgatorio.

\subsection{L'inferno}

L'inferno è composto da settori sempre più stretti. Più lo spazio diminuisce e più i peccati sono immorali
secondo Date.
Principalmente, l'inferno è suddiviso in 3 sezioni.
Dall'alto verso il basso, ci sono gli \textit{incontinenti}, \textit{violenti} e
i \textit{freudolenti}.

\sdefinition{Legge del contrappasso}{
    La \textit{legge del contrappasso} associa una pena
    legata alla colpa.
    Il nesso avviene o per \textit{analogia} o per \textit{opposizione}.
}

%\subsection{Struttura dell'inferno}
%\begin{figure}[h]
%    \centering
%    \includegraphics[width=0.5\textwidth]{./inferno.jpg}
%\end{figure}

\subsection{Struttura del testo}

% https://tikzcd.yichuanshen.de/#N4Igdg9gJgpgziAXAbVABwnAlgFyxMJZABgBoBGAXVJADcBDAGwFcYkQAdDnGADx2ABjemDyCAFjAC+IKaXSZc+QinIVqdJq3Zce-XMAAKzAE4Bzejggn8MuQux4CRNcQ0MWbRJ2588AgEkwADMYE0g7eRAMR2UXUgAmdy0vHz1-I3oTeigsOAhIhyVnFATSNxoPbW9dP2ByAGoAZiaAAmFRLELoxScVZDKqSpSdX35gFvaRPG6Y4v6ypOHPUfSJto6Z2Q0YKDN4IlBgkwgAWyQykCskABYaSRz2HAB3CAeoBHsQY7OkNSuIEgAKz3GCPbwvN5gj6yKI-c6IJo0a6IABsoPBV1e70+cJOCLIAL+X3hSEJKMuACMYGAoEgmsQSfiycjAYiaNTaUgALQMqSUKRAA
\begin{center}
\begin{tikzcd}
    & \textit{Inferno} \arrow[r, two heads]    & \text{1+33 canti} \\
    \text{Cantiche} \arrow[r] \arrow[ru, bend left] \arrow[rd, bend right] & \textit{Purgatorio} \arrow[r, two heads] & \text{33 canti}   \\
    & \textit{Paradiso} \arrow[r, two heads]   & \text{33 canti}  
\end{tikzcd}
\end{center}

\subsection{Lucifero}

Lucifero viene rappresentato come un grande orrenda cretura con 3 bocche.
In ogni bocca mastica per l'eternità i 3 peccatori più grandi.
Al centro Giuda, mentre ai lati Bruto e Cassio.

\pagebreak

\subsection{Inferno}

\subsubsection{Inferno, Canto I}

Il primo canto dell'inferno fa da proemio a tutta la Commedia.

\begin{center}
    \textit{Nel mezzo del cammin di nostra vita} \\
    \textit{mi ritrovai per una selva oscura,} \\
    \textit{ché la diritta via era smarrita.}
\end{center}

la vita media è di 70 anni. Questo è un dato biblico e non il valore della vita media.
Dante è nato nel 1265, per cui 1265+35=1300. Il percorso inizia infatti esattamente nel 1300.
Questo dato viene anche confermato in \textit{Inferno XXI, 112-114}.

\sdefinition{Giubileo}{
    Il \textit{giubileo} è un anno di assoluzione collettica di peccati.
}
Questa data è quella del primo Giubileo, indetto dal Papa Bonifacio VIII.

\sdefinition{Allegoria}{
    Figura retorica per mezzo della quale l'autore esprime e il lettore ravvisa un significato riposto,
    diverso da quello letterale.
}
La selva oscura, dove non vi è luce, rappresenta una condizione di peccato.
La \textit{diritta via} è quella che conduce a Dio.
Essa è smarrita, ma può essere appunto ritrovata.
\\
Dante non specifica il tipo di peccato, questo è dato dal fatto che Dante rappresenta allegoricamente
l'interno dell'umanità (nel 1300), per cui il peccato di tutti gli uomini in quel periodo.

\begin{center}
    \textit{Ahi quanto a dir qual era è cosa dura} \\
    \textit{esta selva \textbf{selvaggia} e \textbf{aspra} e \textbf{forte}} \\
    \textit{che nel pensier rinova la paura!}
\end{center}

I tre aggettivi; selvaggia (disumano), aspra (fitta) e forte (da cui è difficile uscire)
sono disposti a climax.

\begin{center}
    \textit{Tant' è amara che poco è più morte;} \\
    \textit{ma per trattar del ben ch'i' vi trovai,} \\
    \textit{dirò de l'altre cose ch'i' v'ho scorte.}
\end{center}

La morte, che è la cosa più terribile che ci sia, lo è solamente poco più della selva.
\\
I due verbi sui quali si chiude \textit{Vita Nova},
\textbf{dire} e \textbf{trattare}, si ritrovano all'inizio della \textit{Commedia}.
In \textit{Vita Nova} questi verbi si riferiscono all'io poetico, mentre all'inizio della \textit{Commedia}
sono riferiti al \textbf{bene} e ad \textbf{altre cose}.
Il bene si riferisce alla salvezza (Dio), mentre altre cose si riferisce a tutto ciò che trovò durante il viaggio.
La parola \textit{vi} si riferisce probabilmente a tutto il viaggio compiuto da Dante.

\begin{center}
    \textit{Io non so ben ridir com' i' v'intrai,} \\
    \textit{tant' era pien di sonno a quel punto} \\
    \textit{che la verace via abbandonai.}
\end{center}

Il sonno rappresenta in senso allegorico il sonno della coscienza, che porta al peccato, ossia la selva.

\snote{Funzioni di Dante}{
    Date ha diverse funzioni che si intrecciano nel racconto.
    \begin{itemize}
        \item Dante personaggio, pellegrino che compie il viaggio
        \item Dante allegoria per tutta l'umanità
        \item Dante poeta fiorentino
    \end{itemize}
}

\begin{center}
    \textit{Ma poi ch'i' fui al piè d'un colle giunto,} \\
    \textit{là dove terminava quella valle} \\
    \textit{che m'avea di paura il cor compunto,}
\end{center}

\begin{center}
    \textit{guardai in alto e vidi le sue spalle} \\
    \textit{vestite già de' raggi del pianeta} \\
    \textit{che mena dritto altrui per ogne calle.}
\end{center}

Le spalle del colle sono il punto in cui la collina si piega.
Questa perifrasi indica semplicemente che la collina era illuminata dalla luce solare.
In alto vi è la luce divina, mentre in basso c'è il buio del peccato.
Il collo rappresenta infatti il percorso difficile; è molto più facile
cadere all'inferno che giungere a Dio. \\
Il gesto di guardare in alto indica un progressivo distaccarsi dal peccato.

\begin{center}
    \textit{Allor fu la paura un poco queta,} \\
    \textit{che nel lago del cor m'era durata} \\
    \textit{la notte ch'i' passai con tanta pieta.}
\end{center}

\begin{center}
    \textit{E come quei che con \textbf{lena affannata},} \\
    \textit{uscito fuor del pelago a la riva,} \\
    \textit{si volge a l'acqua perigliosa e guata,}
\end{center}

% pelago = mare
Il verbo \textbf{guatare} significa guardare con partecipazione, spesso paura.
Questa similitudine mette in confronto un naufrago che scampa il pericolo dell'acqua,
che come Dante scampa dalla selva e si gira a guardarla, con un sentimento di sollievo.
Il corpo di Dante è fermo, ma il suo animo è ancora spaventato e vorrebbe continuare a scappare.

Questo indica anche che indugiare nel peccato, come indugiare nella selva o nelle acque, porta alla morte.

\begin{center}
    \textit{così l'animo mio, ch'ancor fuggiva,} \\
    \textit{si volse a retro a rimirar lo passo} \\
    \textit{che non lasciò già mai persona viva.}
\end{center}

\begin{center}
    \textit{Poi ch'èi posato un poco il corpo lasso,} \\
    \textit{ripresi via per la piaggia diserta,} \\
    \textit{sì che 'l piè fermo sempre era 'l più basso.}
\end{center}

La piaggia è un leggero pendio che non è ancora l'effettiva salita.
\\
Il terzo verso, dove un piede è sempre più basso dell'altro,
implica che ci sia ancora una zavorra che lo mantenga vicino al peccato (alla selva).

\begin{center}
    \textit{Ed ecco, quasi al cominciar de l'erta,} \\
    \textit{una lonza leggiera e presta molto,} \\
    \textit{che di pel macolato era coverta;}
\end{center}

Dante incontra la prima delle tre fiere, la lonza.
\sdefinition{Lussuria}{
    La lussuria è un vizio inteso come l'abbandono alle proprie passioni o anche a divertimenti di natura generica, senza il controllo da parte della nostra ragione e della nostra morale.
}
La lonza è leggera (veloce, agile). Essa rappresenta infatti la lussuria.

\begin{center}
    \textit{e non mi si partia dinanzi al volto,} \\
    \textit{anzi 'mpediva tanto il mio cammino,} \\
    \textit{ch'i' fui per ritornar più volte vòlto.}
\end{center}

\begin{center}
    \textit{Temp' era dal principio del mattino,} \\
    \textit{e 'l sol montava 'n sù con quelle stelle} \\
    \textit{ch'eran con lui quando l'amor divino}
\end{center}

Il tempo è il mattino, e la stazione è la primavera.
Secondo la \textit{Genesi} il mondo è stato creato di primavera.

\begin{center}
    \textit{mosse di prima quelle cose belle;} \\
    \textit{sì ch'a bene sperar m'era cagione} \\
    \textit{di quella fiera a la gaetta pelle}
\end{center}

\begin{center}
    \textit{l'ora del tempo e la dolce stagione;} \\
    \textit{ma non sì che paura non mi desse} \\
    \textit{la vista che m'apparve d'un leone.}
\end{center}

Dante ritrova speranza pensando che il mattino di primavera sia un momento propizio di inizio.
\\
Dante incontro la seconda delle fiere, il leone.

\begin{center}
    \textit{Questi parea che contra me venisse} \\
    \textit{con la test' alta e con rabbiosa fame,} \\
    \textit{sì che parea che l'aere ne tremesse.}
\end{center}

\sdefinition{Superbia}{
    Radicata convinzione della propria superiorità (reale o presunta) che si traduce in atteggiamenti di orgoglioso distacco o anche di ostentato disprezzo verso gli altr
}
Il leone rappresenta la superbia.

\begin{center}
    \textit{Ed una lupa, che di tutte brame} \\
    \textit{sembiava carca ne la sua magrezza,} \\
    \textit{e molte genti fé già viver grame,}
\end{center}

Immediatamente Date incontro anche la terza fiera, la lupa.
\sdefinition{Avarizia (antica)}{
    La brama di possessi materialistici.
}
La lupa è molto magra. Essa rappresenta l'avariazia, la fame insaziabile e la brama di possessi materiali.

\begin{center}
    \textit{questa mi porse tanto di gravezza} \\
    \textit{con la paura ch'uscia di sua vista,} \\
    \textit{ch'io perdei la speranza de l'altezza.}
\end{center}

Questa gravezza (peso) simboleggia il ritorno verso la sede, perdendo la speranza di salire.

\snote{Alternarsi della speranza}{
    Dante viene attraversato da un continuo alternarsi fra speranza e disperazione.
}

\begin{center}
    \textit{E qual è quei che volontieri acquista,} \\
    \textit{e giugne 'l tempo che perder lo face,} \\
    \textit{che 'n tutti suoi pensier piange e s'attrista;}
\end{center}

Questa similitudine si riferisce agli avari (oppure potrebbe riferisci ai giocatori d'azzardo).
Coloro che affidano la propria felicità ai beni materiali, e si disperano quando perdono tuto.


\begin{center}
    \textit{tal mi fece la bestia sanza pace,} \\
    \textit{che, venendomi 'ncontro, a poco a poco} \\
    \textit{mi ripigneva là dove 'l sol tace.}
\end{center}

La lupa faceva provare a Dante la stessa sensazione dell'ultima terzina,
riportandolo verso la selva (dove il sole non splende).

È presente una sinestesia (dove 'l sol tace).

\begin{center}
    \textit{Mentre ch'i' rovinava in basso loco,} \\
    \textit{dinanzi a li occhi mi si fu offerto} \\
    \textit{chi per lungo silenzio parea fioco.}
\end{center}

Mentre Dante rotolava verso il basso, incontra \textbf{Virgilio}.
La sua voce era bassa perché non aveva parlato per molto tempo.
Questo indica anche che la sua parola non veniva ascoltata da molto,
esso rappresenta infatti la ragione umana.

\sproposition{Virgilio}{
    \textit{Virgilio} fu un poeta
    vissuto tra il 70 a.C e il 19 a.C.
}

\begin{center}
    \textit{Quando vidi costui nel gran diserto,} \\
    \textit{«Miserere di me», gridai a lui,} \\
    \textit{«qual che tu sii, od ombra od omo certo!».}
\end{center}

Questa è la prima volta che qualcuno parla.

Dante chiede \quotes{Abbi pietà di me. Chiunque tu sia, anima o uomo}.

\begin{center}
    \textit{Rispuosemi: «Non omo, omo già fui,} \\
    \textit{e li parenti miei furon lombardi,} \\
    \textit{mantoani per patrïa ambedui.}
\end{center}

Nel medioevo le persone si presentavano con la loro provenienza geografica.
Ciò indica il nome della propria famiglia e l'appartenenza politica.

La Lombardia era tutta l'Italia del Nord.
I genitori erano mantovani.

\begin{center}
    \textit{Nacqui sub Iulio, ancor che fosse tardi,} \\
    \textit{e vissi a Roma sotto 'l buono Augusto} \\
    \textit{nel tempo de li dèi falsi e bugiardi.}
\end{center}

L'anima è nata durante il periodo di Giulio Cesare, ma visse
a Roma sotto Augusto, a seguito della morte di Cesare nel 44 a.C.

Virgilio ha vissuto in un periodo di Dei pagani (siccome Cristo non era ancora nato).

\begin{center}
    \textit{Poeta fui, e cantai di quel giusto} \\
    \textit{figliuol d'Anchise che venne di Troia,} \\
    \textit{poi che 'l superbo Ilïón fu combusto.}
\end{center}

Virgilio celebrò di Enea (Eneide) dopo che la fortezza fu bruciata.
Qui termina la presentazione.

È importante notare che la salvezza di Dante deriva da un poeta.
La poesia era cruciale nel mondo medievale.

\begin{center}
    \textit{Ma tu perché ritorni a tanta noia?} \\
    \textit{perché non sali il dilettoso monte} \\
    \textit{ch'è principio e cagion di tutta gioia?».}
\end{center}

In italiano antico la noia indica tormento.

\hr

Nonostante ci si aspetterebbe la risposta di Dante, esso
riconosce Virgilio e lo elogia con le seguenti 3 terzine:

\begin{center}
    \textit{«Or se' tu quel Virgilio e quella fonte} \\
    \textit{che spandi di parlar sì largo fiume?»,} \\
    \textit{rispuos' io lui con \textbf{vergognosa} fronte.}
\end{center}

Dante si rivolge a Virgilio con vergogna, sentimento di deferenza e rispetto.

\begin{center}
    \textit{«O de li altri poeti onore e lume,} \\
    \textit{vagliami 'l lungo studio e 'l grande amore} \\
    \textit{che m'ha fatto cercar lo tuo volume.}
\end{center}

Dante dice di avere studiato la sua opera, e dichiara un debito poetico.
L'amore poetico di Dante l'ha portato a studiare (a memoria) l'Eneide.
Infatti, molte espressioni nella Commedia sono riprese dall'Eneide.

\begin{center}
    \textit{Tu se' lo mio maestro e 'l mio autore,} \\
    \textit{tu se' solo colui da cu' io tolsi} \\
    \textit{lo bello stilo che m'ha fatto onore.}
\end{center}

\hr

\begin{center}
    \textit{Vedi la bestia per cu' io mi volsi;} \\
    \textit{aiutami da lei, famoso saggio,} \\
    \textit{ch'ella mi fa tremar le vene e i polsi».}
\end{center}

La lonza e il leone non vengono nemmeno più nominati, \quotes{la bestia} è quella più difficile.

\begin{center}
    \textit{«A te convien tenere altro vïaggio»,} \\
    \textit{rispuose, poi che lagrimar mi vide,} \\
    \textit{«se vuo' campar d'esto loco selvaggio;}
\end{center}

Il primo verso di questa terzina è quello più importante di tutto il canto.
Virgilio risponde indicando un altro percorso da compiere

\begin{center}
    \textit{ché questa bestia, per la qual tu gride,} \\
    \textit{non lascia altrui passar per la sua via,} \\
    \textit{ma tanto lo 'mpedisce che l'uccide;}
\end{center}

la lupa non lascia passare \textit{nessuno}.
Indugiare qui significa morire.

\begin{center}
    \textit{e ha natura sì malvagia e ria,} \\
    \textit{che mai non empie la bramosa voglia,} \\
    \textit{e dopo 'l pasto ha più fame che pria.}
\end{center}

Viriglio descrive ulteriorment la lupa.
La lupa ha ancora più fame dopo aver mangiato, questa è la cupidigia.

\begin{center}
    \textit{Molti son li animali a cui s'ammoglia,} \\
    \textit{e più saranno ancora, infin che 'l veltro} \\
    \textit{verrà, che la farà morir con doglia.}
\end{center}

Molte sono le vittime di questo peccato, ma Virgilio profetizza che il veltro sia l'unico
a poterla superare.
Possiamo capire che questa sia una profezia dal tempo futuro e linguaggio enigmatico.

\begin{center}
    \textit{Questi non ciberà terra né peltro,} \\
    \textit{ma sapïenza, amore e virtute,} \\
    \textit{e sua nazion sarà tra feltro e feltro.}
\end{center}

%  TODO Dieresi - spezza le due vocali

Il veltro non si ciberà nè di terra nè di peltro (lega metallica delle monete).
Non avrà quindi fame di ricchezza materiale.
Invece, si ciberà di sapienza, amore e virtù (i 3 attributi della trinità).

Questo personaggio nascerà fra \textit{feltro} e \textit{feltro} (un panno, tessuto).
la prima interpretazione è quella di interpretare il feltro come un panno vile.

\sdefinition{Veltro}{
    Il \textit{veltro} è un cane da caccia.
    Nella letteratura italiana, rappresenta un'azione di riforma,
    evidentemente promossa da Dio, che perseguiti la cupidigia in tutte le sue forme
    ristabilendo in tutto il mondo ordine e giustizia. 
}

La lupa potrebbe essere sconfitta da un uomo di quella chiesa (probabilmente dei Francescana),
che si occupa dei malati ed è umile.

Un'altra interpretazione vede il \textit{feltro} come il cielo,
mentre un'altra lo collega alla collocazione geografia di Verona (la quale si
situa tra Feltre e Montefeltro), per cui nascerà a Verona.

Un'ulteriore interpretazione, quella più accreditata, dice che esso nascerà da un'elezione imperiale
(probabilmente Arrigo VIII), siccome l'urna veniva foderata di feltro all'interno.

\begin{center}
    \textit{Di quella umile Italia fia salute} \\
    \textit{per cui morì la vergine Cammilla,} \\
    \textit{Eurialo e Turno e Niso di ferute.}
\end{center}

\textbf{\color{red}nota:} fia = sarà. \\
Virgilio sta dicendo chde il veltro sarà la salvezza dell'Italia.

\begin{center}
    \textit{Questi la caccerà per ogne villa,} \\
    \textit{fin che l'avrà rimessa ne lo 'nferno,} \\
    \textit{là onde 'nvidia prima dipartilla.}
\end{center}

Questa terzina termina la profezia.
Il veltro sconfiggerà la lupa cacciandola ovunque fino all'inferno.
L'ultimo verso sembrerebbe indicare il primo momento in cui la lupa è stata scagliata
fra gli uomini, a seguito dell'invidia del demonio nei confronti di Dio.

\begin{center}
    \textit{Ond' io per lo tuo me' penso e discerno} \\
    \textit{che tu mi segui, e io sarò tua guida,} \\
    \textit{e trarrotti di qui per loco etterno;}
\end{center}

Virgilio dice che per il meglio di Dante, è auspicabile che lui lo segua.
Questa è infatti la saggezza di Virgilio.
Dante verrà salvato e portato via attraverso un luogo eterno (l'inferno).

\begin{center}
    \textit{ove udirai le disperate strida,} \\
    \textit{vedrai li antichi spiriti \textbf{dolenti},} \\
    \textit{ch'a la seconda morte ciascun grida;}
\end{center}

Queste terzine descrivono quindi l'inferno, dove Dante passerà.
Esso viene rappresentato come pieno di anime dannate.

La prima morta è quella fisica, mentre la seconda morte è quella dell'anima che
diventa dannata.

\begin{center}
    \textit{e vederai color che son \textbf{contenti}} \\
    \textit{nel foco, perché speran di venire} \\
    \textit{quando che sia a le beate genti.}
\end{center}

Le anime dannate sono distrutte nel dolore perché sanno che quella è la loro
fine eterna, mentre le anime nel purgatorio sono contente di scontare la propria pena,
perché sanno che essa avrà una fine (fino all'Apocalisse).

Questo può essere sintetizzato dalla rima \quotes{dolenti}:\quotes{contenti}.

\begin{center}
    \textit{A le quai poi se tu vorrai salire,} \\
    \textit{anima fia a ciò più di me degna:} \\
    \textit{con lei ti lascerò nel mio partire;}
\end{center}

Se Dante vorrà salire fra le genti beate (paradiso), non potrà farlo con Virgilio
ma con un'altra persona (Beatrice).

\begin{center}
    \textit{A le quai poi \textbf{se} tu vorrai salire,} \\
    \textit{anima fia a ciò più di me degna:} \\
    \textit{con lei ti lascerò nel mio partire;}
\end{center}

Dante avrà la scelta di percorrere anche i cieli. Non è infatti necessario
farlo per essere salvi. Dante sarà salvo dopo aver percorso il purgatorio.

\begin{center}
    \textit{ché quello imperador che là sù regna,} \\
    \textit{perch' i' fu' ribellante a la sua legge,} \\
    \textit{non vuol che 'n sua città per me si vegna.}
\end{center}

Il motivo per cui Dante dovrà essere seguire da un'altra anima
è perché Virgilio è un dannato (non è battezzato). 
Virgilio non può nominare direttamente Dio essendo tale.

% virtù cardinali e teologate

\begin{center}
    \textit{In tutte parti impera e quivi regge;} \\
    \textit{quivi è la sua città e l'alto seggio:} \\
    \textit{oh felice colui cu' ivi elegge!».}
\end{center}

Beato colui che può accedere in paradiso.

\begin{center}
    \textit{E io a lui: «Poeta, io ti richeggio} \\
    \textit{per quello Dio che tu non conoscesti,} \\
    \textit{a ciò ch'io fugga questo male e peggio,}
\end{center}

A differenza di Virgilio, Dante può pronunciare il nome di Dio.
Nel nome del Dio che purtroppo Virgilio non potrà conoscere, Dante accetta.

\begin{center}
    \textit{che tu mi meni là dov' or dicesti,} \\
    \textit{sì ch'io veggia la porta di san Pietro} \\
    \textit{e color cui tu fai cotanto mesti».}
\end{center}

Dante ripassa i posti che traverserà con Virgilio, ma in ordine opposto.
La porta di san Pietro rappresenta il purgatorio, mentre le anime dannate
sono nell'inferno.

\begin{center}
    \textit{Allor si mosse, e io li tenni dietro.}
\end{center}

Come il canto comincia con un cammino (metaforico), viene terminato
con un cammino (fisico).

\pagebreak

\subsubsection{Inferno, Canto II}

Il secondo canto serve da proemio al libro dell'Inferno.
Tutto il secondo canto si svolge nel medesimo punto dove il primo canto termina.
Infatti, Dante e Virgilio non si muovono.

\sdefinition{Invocazione alle Muse}{
    Nel ricevere e far suo il tradizionale uso retorico d'invocare le Muse, quando più arduo si presenti l'impegno dell'arte.
}
Dante invoca le Muse, ingegno e memoria per quello che sta per scrivere.

A dante giungono dei dubbi, ossia quale sia lo scopo del suo viaggio
(la risposta verrà data in Paradiso XVII) e chi gli permetta di fare tale.
Perché Dante ha il permesso di compiere questo viaggio?
\\
Lo stesso viaggio è stato compiuto solamente da San Paolo (Nuovo Testamento)
e Enea (Eneide), uno ai cieli e uno agli inferi.

Virgilio risponde alla seconda domanda mediante le seguenti affermazioni.
È stata Beatrice ad avvisarlo che Dante fosse in difficoltà.
La Madonna ha visto la difficoltà di Dante, l'ha detto a Santa Lucia, che va da Beatrice,
la quale scende da Virgilio.
Per cui, il viaggio è permesso da 3 donne benedette.
\begin{center}
    \begin{tikzcd}
        \text{Madonna} \arrow[r] & \text{Santa Lucia} \arrow[r] & \text{Beatrice} \arrow[d] \\
        & \text{Dante}                 & \text{Virgilio} \arrow[l]
    \end{tikzcd}
\end{center}
Il motivo per cui Beatrice ha scelto Virgilio è per la sua ragione e uso della parola.

Beatrice si sente infatti in Debito con Dante per la sua gratitudine (\textit{Vita Nova}).

\pagebreak

\subsubsection{Inferno, Canto III}

\hr

Le seguenti 3 terzine sono le incisioni sulla porta dell'inferno.

\begin{center}
    \textit{"Per me si va ne la città \textbf{dolente},} \\
    \textit{per me si va ne l'etterno \textbf{dolore},} \\
    \textit{per me si va tra la \textbf{perduta gente}.}
\end{center}

La porta parla al singolare con un'anafora di \quotes{Per me si va}.
La prima terzina esprime il dolore che è presente nel posto.

\begin{center}
    \textit{Giustizia mosse il mio alto fattore;} \\
    \textit{fecemi la divina podestate,} \\
    \textit{la somma sapïenza e 'l primo amore.}
\end{center}

Questi sono i tre attributi della trinità (Io, inferno, sono fatto da Dio).
Tutto quello che si vedrà è quindi un luogo giusto, perché Dio è giusto per definizione.
Nell'inferno manca quindi arbitrio.

\begin{center}
    \textit{Dinanzi a me non fuor cose create} \\
    \textit{se non etterne, e io etterno duro.} \\
    \textit{Lasciate ogne speranza, voi ch'intrate".}
\end{center}

L'inferno è sempiterno.

\hr

\begin{center}
    \textit{Queste parole di colore oscuro} \\
    \textit{vid'ïo scritte al sommo d'una porta;} \\
    \textit{per ch'io: "Maestro, il senso lor m'è duro".}
\end{center}

Le scritte sono di colore nero, scure e hanno la funzione opposte delle scritte
che c'erano nei portoni delle chiese medievali, le quali invitavano i fedeli ad entrare.

\begin{center}
    \textit{Ed elli a me, come persona accorta:} \\
    \textit{"Qui si convien lasciare ogne sospetto;} \\
    \textit{ogne viltà convien che qui sia morta.}
\end{center}

Virgilio dice a Dante che deve lasciare ogni sospetto (esitazione).
L'animità non ha più il tempo di esitare.

Dante è un uomo con le sue debolezze ma che necessita umanamente di un uomo che lo aiuti con la sua paura.

\begin{center}
    \textit{Noi siam venuti al loco ov'i' t' ho detto} \\
    \textit{che tu vedrai le genti dolorose} \\
    \textit{c' hanno perduto il ben de l'intelletto". }
\end{center}

\begin{center}
    \textit{E poi che la sua mano a la mia puose} \\
    \textit{con lieto volto, ond'io mi confortai,} \\
    \textit{mi mise dentro a le segrete cose. }
\end{center}

Questo ultimo verso è l'ultimo alla luce del sole.

\begin{center}
    \textit{Quivi \textbf{sospiri}, \textbf{pianti} e \textbf{alti guai}} \\
    \textit{risonavan per l'aere sanza stelle,} \\
    \textit{per ch'io al cominciar ne lagrimai. }
\end{center}

Le prime percezioni sono acustiche siccome Dante non vede quasi nulla (area senza stelle).
Per tutto l'inferno Dante proverà compassione dei dolenti.

Climax: \textbf{sospiri}, \textbf{pianti} e \textbf{alti guai}.

\begin{center}
    \textit{Diverse \textbf{lingue}, orribili \textbf{favelle},} \\
    \textit{\textbf{parole} di dolore, \textbf{accenti} d'ira,} \\
    \textit{voci alte e fioche, e suon di man con elle }
\end{center}

Anti-climax di dettaglio: \textbf{lingue}, \textbf{favelle}, \textbf{parole}, \textbf{accenti} (suoni).

\begin{center}
    \textit{facevano un tumulto, il qual s'aggira} \\
    \textit{sempre in quell'aura sanza tempo tinta,} \\
    \textit{come la rena quando turbo spira.}
\end{center}

Tutti i suoini giungono a Dante come se fossero un vortice (tromba d'aria) di sabbia.

\begin{center}
    \textit{E io ch'avea d'error la testa cinta,} \\
    \textit{dissi: "Maestro, che è quel ch'i' odo?} \\
    \textit{e che gent'è che par nel duol sì vinta?". }
\end{center}

\begin{center}
    \textit{Ed elli a me: "Questo misero modo} \\
    \textit{tegnon l'anime triste di coloro} \\
    \textit{che visser \textbf{sanza 'nfamia e sanza lodo}. }
\end{center}

\sdefinition{I pusillanimi}{
    I \textit{pusillanimi} sono coloro che rifiutano la loro identità di uomo,
    sono timidi e non hanno coraggio e determinazione, per cui non esercitano il libero arbitrio.
}

Virgilio risponde che questo misero modo di lamentarsi appartiene
alle anime dei pusillanimi (ignavi), ossia coloro che non hanno preso decisioni
nella loro vita, non hanno commesso nè bene (lodo) nè male (infamia).

La loro colpa è quella di non aver esercitato il libero arbitrio di Dio,
per cui rinunciare alla identità più profonda di uomo e vivere come un animale.

\begin{center}
    \textit{Mischiate sono a quel cattivo coro} \\
    \textit{de li angeli che \textbf{non furon ribelli}} \\
    \textit{\textbf{né fur fedeli} a Dio, ma per sé fuoro.}
\end{center}

Vi è un gruppo di angeli che sta a sè (non sta nè con Dio nè con il Demonio), angeli che non si schierarono.

Questa categoria non è riconosciuta dalle Scrittura ma riconosciuta dalla cultura popolare.

\begin{center}
    \textit{\textbf{Caccianli i ciel} per non esser men belli,} \\
    \textit{\textbf{né lo profondo inferno li riceve},} \\
    \textit{ch'alcuna gloria i rei avrebber d'elli».}
\end{center}

Virgilio smette di parlare dicendo che queste anime sono cacciate dai cieli
ma nemmeno nell'inferno profondo. Ribadisce 3 volte che non sono nè da un lato nè dall'altro.

\begin{center}
    \textit{E io: «Maestro, che è tanto greve} \\
    \textit{a lor che lamentar li fa sì forte?».} \\
    \textit{Rispuose: «Dicerolti molto breve.}
\end{center}

\begin{center}
    \textit{Questi non hanno speranza di morte,} \\
    \textit{e la lor cieca vita è tanto bassa,} \\
    \textit{che 'nvidïosi son d'ogne altra sorte.}
\end{center}

Le anime del profondo inferno, potrebbero vantarsi con i pusillani perché loro non hanno un peccato.
Un'anima dannata che vede qualcuno che non ha commesso il male nella sua stessa posizione,
potrebbe sminuire i propri peccati e fare meno importanza alla propria colpa.
Questo è il motivo per cui i pusillani sono espulsi.

Virgilio ritorna una spiegazione molto breve, come se loro non si meritassero nemmeno
più parole. Tutte le anime sperano di essere annichilite, ma questi
hanno una vita cieca tanto bassa e piccola che sono invidiosi di tutti.

\pagebreak

\begin{center}
    \textit{Fama di loro il mondo esser non lassa;} \\
    \textit{misericordia e giustizia li sdegna:} \\
    \textit{non ragioniam di lor, ma guarda e passa».}
\end{center}

Queste anime non mancheranno al mondo e non verranno ricordati.
Dante e Virgilio hanno un grosso disprezzo verso questi spiriti.
Non meritano nè parola nè tempo.

\begin{center}
    \textit{E io, che riguardai, vidi una 'nsegna} \\
    \textit{che girando correva tanto ratta,} \\
    \textit{che d'ogne posa mi parea indegna;}
\end{center}

Dante vede una bandiera che corre molto rapida, la quale viene seguita
da una massa di gente. È sorpreso che ci siano tanti pusillanimi.

\begin{center}
    \textit{che girando correva tanto ratta,} \\
    \textit{di gente, ch'i' non averei creduto} \\
    \textit{che morte tanta n'avesse disfatta.}
\end{center}

Viene quindi mostrata la prima relazione fra pena e colpa per contrappasso,
in questo caso, per opposizione. Infatti, le anime devono sepre seguire il vessillo (simbolo dello schieramento).
Così come in vita non si sono mai schierati, ora devono schierarsi per sempre.

\begin{center}
    \textit{Poscia ch'io v'ebbi alcun riconosciuto,} \\
    \textit{vidi e conobbi l'ombra di colui} \\
    \textit{che fece per viltade il gran rifiuto.}
\end{center}

Dante ne riconosce alcuni, in particolare l'anima di colui
che per pusillanimità fece un gran rifiuto.

Si riferisce probabilmente al papa Celestino V, il quale si è dimesso
dopo qualche mese. Le sue dimissioni fecero eleggere Bonifacio VIII, il quale manderà Dante in esilio.
A supporto di questa tesi vi è la frase "vidi e conobbi", il personaggio è quindi un contemporaneo
di Dante. Inoltre, già tutti i commentatori antichi, i quali erano vinici a quest'epoca,
hanno riferito questo papa.

Dante avrebbe potuto scrivere il canto prima che il papa diventasse santo
perché forse non avrebbe messo un santo all'inferno.
D'altro canto non sarebbe un eresia farlo, in quanto è una scelta della chiesa.

\begin{center}
    \textit{Incontanente intesi e certo fui} \\
    \textit{che questa era la setta d'i cattivi,} \\
    \textit{a Dio spiacenti e a' nemici sui.}
\end{center}

\begin{center}
    \textit{Questi sciaurati, che mai non fur vivi,} \\
    \textit{erano ignudi e stimolati molto} \\
    \textit{da mosconi e da vespe ch'eran ivi.}
\end{center}

Questa è la seconda parte della pena: erano nudi (come tutte le anime)
e punti da vespe e mosconi.

\begin{center}
    \textit{Elle rigavan lor di sangue il volto,} \\
    \textit{che, mischiato di lagrime, a' lor piedi} \\
    \textit{da \textbf{fastidiosi} vermi era ricolto.}
\end{center}

Questo è un contrappasso per analogia. Così come in vita sono stati insignificanti,
ora sono tormentati piccoli insetti insegnificanti come loro (mosconi vespe e vermi).

Lacrime e sangue indicano uno sforzo, quello sforzo che loro non hanno mai compiuto.

\begin{center}
    \textit{E poi ch'a riguardar oltre mi diedi,} \\
    \textit{vidi genti a la riva d'un gran fiume;} \\
    \textit{per ch'io dissi: «Maestro, or mi concedi}
\end{center}

\begin{center}
    \textit{ch'i' sappia quali sono, e qual costume} \\
    \textit{le fa di trapassar parer sì pronte,} \\
    \textit{com' i' discerno per lo fioco lume».}
\end{center}

Guardando in un'altra direzione, oltre un fiume, vede molte genti e chiede a Virgilio
chi sono e perché sembrino (dalla poca luca) desiderose di attraversare la riva.

\begin{center}
    \textit{Ed elli a me: «Le cose ti fier conte} \\
    \textit{quando noi fermerem li nostri passi} \\
    \textit{su la trista riviera d'Acheronte».}
\end{center}

Virgilio indica che gli darà la risposta quando arriveranno là.
Il primo fiume infernale si chiama Acheronte.

\begin{center}
    \textit{Allor con li occhi vergognosi e bassi,} \\
    \textit{temendo no 'l mio dir li fosse grave,} \\
    \textit{infino al fiume del parlar mi trassi.}
\end{center}

Dante, temendo di essere stato inopportuno, fa silenzio sino all'arrivo al fiume.

\begin{center}
    \textit{Ed ecco verso noi venir per nave} \\
    \textit{un vecchio, bianco per antico pelo,} \\
    \textit{gridando: «Guai a voi, anime prave!}
\end{center}

Giunge verso Dante un vecchio con barba e capelli bianchi su una nave che grida.
Esso è Caronte ed è il traghettatore di anime.

\begin{center}
    \textit{\textbf{Non isperate} mai veder lo cielo:} \\
    \textit{i' vegno per menarvi a l'altra riva} \\
    \textit{ne le tenebre \textbf{etterne}, in caldo e 'n gelo.}
\end{center}

Le parole di Caronte replicano l'iscrizione della porta dell'inferno.
Il personaggio è già presente con il medesimo ruolo nell'Eneide.

\begin{center}
    \textit{E tu che se' costì, anima viva,} \\
    \textit{pàrtiti da cotesti che son morti».} \\
    \textit{Ma poi che vide ch'io non mi partiva,}
\end{center}

Caronte si ritrova davanti un vivo (Dante) e gli dice che lui non è un dannato, e che quindi se ne deve andare.

\begin{center}
    \textit{disse: «Per altra via, per altri porti} \\
    \textit{verrai a piaggia, non qui, per passare:} \\
    \textit{più lieve legno convien che ti porti».}
\end{center}

Una volta visto che Dante non si muoveva, gli dice
"[Quando sarà il tuo momento], farai un altro percorso su una barca più leggera".
Per cui, non è un'anima dell'inferno.

\begin{center}
    \textit{E 'l duca lui: «Caron, non ti crucciare:} \\
    \textit{vuolsi così colà dove si puote} \\
    \textit{ciò che si vuole, e più non dimandare».}
\end{center}

Virgilio interviene e dice a Caronte di far passare Dante
siccome il suo viaggio è voluto da Dio, e per cui Caronte non deve interferire.

\begin{center}
    \textit{Quinci fuor quete le lanose gote} \\
    \textit{al nocchier de la livida palude,} \\
    \textit{che 'ntorno a li occhi avea di fiamme rote.}
\end{center}

% nocchier = tragetatore
Caronte ha delle fiamme attorno agli occhi. Il suo ritratto temrina qui, e lui si tranquillizza.

\begin{center}
    \textit{Ma quell' anime, ch'eran lasse e nude,} \\
    \textit{cangiar colore e dibattero i denti,} \\
    \textit{ratto che 'nteser le parole crude.}
\end{center}

A differenza di Dante, le anime dannate sanno che quelle parole erano rivolte a loro.

\begin{center}
    \textit{Bestemmiavano \textbf{Dio} e lor \textbf{parenti},} \\
    \textit{\textbf{l'umana spezie} e 'l \textbf{loco} e 'l \textbf{tempo} e 'l \textbf{seme}} \\
    \textit{di lor semenza e di lor nascimenti.}
\end{center}

Anti-climax: \textbf{Dio}, \textbf{parenti}, \textbf{l'umana spezie}, \textbf{tempo}
e \textbf{seme}. \\
Le anime maledicono il giorno della loro nascita.

\begin{center}
    \textit{Poi si ritrasser tutte quante insieme,} \\
    \textit{forte piangendo, a la riva malvagia} \\
    \textit{ch'attende ciascun uom che Dio non teme.}
\end{center}

I pusillanimi si trovano nell'antinferno, uno spazio fra la porta e il fiume.
L'inferno vero comincia infatti dopo il fiume.


\begin{center}
    \textit{Caron dimonio, con occhi di bragia} \\
    \textit{loro accennando, tutte le raccoglie;} \\
    \textit{batte col remo qualunque s'adagia.}
\end{center}

Caronte viene anche definito come un demonio con occhi di brace.
Con un cenno raduna tutte le anime e partono in barca.
Caronte picchia con il remo chi anche solo minimanete provi ad adagiarsi.

\begin{center}
    \textit{Come d'autunno si levan le foglie} \\
    \textit{l'una appresso de l'altra, fin che 'l ramo} \\
    \textit{vede a la terra tutte le sue spoglie,}
\end{center}

Come d'autunno un albero vede tutte le sue foglie cadere,

\begin{center}
    \textit{similemente il mal seme d'Adamo} \\
    \textit{gittansi di quel lito ad una ad una,} \\
    \textit{per cenni come augel per suo richiamo.}
\end{center}

similmente, ad una a una, le anime salgono sulla nave di Caronte.
La similitudine presenta anche un segno di rassegnazione.

\begin{center}
    \textit{Così sen vanno su per l'onda bruna,} \\
    \textit{e avanti che sien di là discese,} \\
    \textit{anche di qua nuova schiera s'auna.}
\end{center}

Caronte non fa a tempo a portare le anime dall'altra parte,
che un nuovo gruppo di anime si raggruppa nuovamente.

\hr

Ecco finalmente le due risposte di Virgilio.

\begin{center}
    \textit{«Figliuol mio», disse 'l maestro cortese,} \\
    \textit{«quelli che muoion ne l'ira di Dio} \\
    \textit{tutti convegnon qui d'ogne paese;}
\end{center}

\begin{center}
    \textit{e pronti sono a trapassar lo rio,} \\
    \textit{ché la divina giustizia li sprona,} \\
    \textit{sì che la tema si volve in disio.}
\end{center}

Ogni uomo, pur dannato e peccatore (eccetto i pusillanimi),
conserva la coscienza, e ritiene la pena giusta in quanto riesce a distinguere fra bene e male,
nonostante la prorpia paura.

\begin{center}
    \textit{Quinci non passa mai anima buona;} \\
    \textit{e però, se Caron di te si lagna,} \\
    \textit{ben puoi sapere omai che 'l suo dir suona».}
\end{center}

Il fatto che Caronte non voglia che Dante sia lì, indica che Dante non ci passerà dopo la morte.

\hr

\begin{center}
    \textit{Finito questo, la buia campagna} \\
    \textit{tremò sì forte, che de lo spavento} \\
    \textit{la mente di sudore ancor mi bagna.}
\end{center}

Vi è un terremoto molto spaventoso, che spaventa ancora Dante scrittore.

\begin{center}
    \textit{La terra lagrimosa diede vento,} \\
    \textit{che balenò una luce vermiglia} \\
    \textit{la qual mi vinse ciascun sentimento;}
\end{center}

\begin{center}
    \textit{e caddi come l'uom cui sonno piglia.} \\
\end{center}

Dante sviene dalla grandissima luce intensa dopo il terremoto.

\subsubsection{Inferno, Canto IV}

Attraversato il fiume vi è il Limbo, ossia la zona di Virgilio.

\pagebreak

\subsection{Caronte Virgiliano e Dantesco}

\sproposition{Caronte}{
    \textit{Caronte} è un personaggio che ha il ruolo di tragettare le anime
    oltre il primo fiume dell'inferno.
}

Vi sono delle differenza fra la descrizione di Caronte nell'\textit{Eneide}
e la \textit{Commedia}.

In ambo i casi, Caronte è un vecchio tragettatore che viene descritto come un personaggio con
delle caratteristiche comuni.
Entrambi sono caratterizzati da dei capelli binachi (rr. 299-300, r. 83) e posseggono
degli occhi di fiamme. Dante indica tuttavia quest'ultima caratteristica 2 volte.
Il ruolo del personaggio è il medesimo; Caronte ha ruolo di ostacolo, ferma chiunque voglia passare
(rr. 338-339, rr. 88-89).

Nella \textit{Commedia}, la barca e Caronte stesso è descritto con più precisione
rispetto all'\textit{Eneide}.
\snote{Descrizioni di Dante}{
    Dante usa più verbi che aggettivi nelle varie descrizioni.
}
Dante descrive Caronte principalmente per quello che fa piuttosto che come è fatto.

Il Caronte Dantesto è più aggressivo (urla), sintatticamente le sue frasi sono
più brevi e nette e viene descritto come indemoniato. Infatti, anche la sua entrata in scena
è improvvisa e violenta. Questa caratteristica non è presente nel Caronte virgiliano,
il quale è più pacato, e descritto come un Dio ma meno importante.

La differenza più importante è che il Caronte dell'\textit{Eneide} fa parte di un mondo pagano.
Non vi è ancora la concezione di salvezza dell'anima, la quale deriva dal cristianesimo.

\pagebreak

\part{Il Duecento e il Trecento}

\subsection{Passagio fra il Duecento e Trecento}

Dante (1265-1321) e Boccaccio (1313 - 1375)
si distinguono per le caratteristiche del cambio di secolo.

\begin{itemize}
    \item Fino ai primi anni del secolo, si assiste ad un aumento della popolazione, favorito dall'accresciuto benesseere e dalle migliorate condizioni sanitarie.
    \item I comuni si espandono sempre di più, grazie alla migliori condizioni economiche e all'immigrazione dal contado. La città diviene il vero e proprio centro della vita: civile, sociale, economica, culturale.
    \item L'esempio di Firenze e l'ascesa della borghesia mercantile.
\end{itemize}

Firenze aveva i fiorini ed era economicamente forte.
Molti poeti e scrittori sono appunto provenienti da paesi del genere.

A differenza di Dante, Boccaccia e Petrarca si distaccano dai loro comuni di appartenzenza,
mentre Dante centralizza Firenze nei suoi temi.

La società comincia lentamente a diventare parzialmente laica.
Il mondo passa dalla trascendenza (visione verticale, tutto è subordinato a Dio)
ad un mondo di immanenza (ciò che è sulla terra).
La teologia nelle università rimane la materia primaria e fondamentale. Medicina è l'unica materia autonoma, ma non è possibile sezionare i cadaveri (corpo sacro).

\subsection{Crisi economica e demografica}

La produzione agricola entra progressivamente in crisi man mano che ci si inoltra nel Trecento.
E una decisa inversione di tendenza rispetto al secolo precedente.\\
In questa situazione già deteriorata, la diffusione in tutta Europa dellla peste nera (1347-1350) provocò
un tracollo economico e una vera e propria crisi demografia (la popolazione si riduce di almeno un terzo).

% Todo le altre cose

\pagebreak

\part{Boccaccio (1313 - 1375)}

\section{Decameron}

\subsection{Introduzione}

L'opera è ambientata durante l'epidemia di peste del 1348 a Firenze
e segue un gruppo di dieci giovani aristocratici (sette donne e tre uomini)
che cercano rifugio nella campagna toscana per sfuggire alla peste.
Per passare il tempo, ciascuno di loro racconta una novella ogni giorno,
per un totale di cento novelle in dieci giorni (14 giorni totali).
Queste novelle spaziano in temi e argomenti, toccando la vita, l'amore, l'umorismo,
l'ingiustizia sociale, la morale, la religione e molti altri aspetti della società medievale.

L'opera è strutturata nella seguente maniera:
\begin{itemize}
    \item \textbf{Introduzione e Proemio:} Inizia con una breve introduzione che spiega le circostanze della fuga dei giovani dalla città e il loro soggiorno in campagna. Il proemio presenta anche i temi principali e lo scopo dell'opera.
        Boccaccio parla della propria sofferenza amorosa, e scrive il Decameron come omaggio alle donne (borghesi, sensibili, che non posso distrarsi essendo confinate in casa).
    \item \textbf{Dieci Giornate:} Ogni giornata rappresenta una sezione dell'opera in cui i personaggi raccontano le novelle. Ogni giornata è caratterizzata da un tema generale o un'idea predominante scelta da un re o una regina che viene eletto ogni giorno.
    \item \textbf{Novelle:} Ogni giornata contiene dieci novelle narrate dai personaggi, ciascuna di esse seguendo il tema giornaliero. Le novelle sono scritte in prosa ed esprimono la varietà delle esperienze umane.
    \item \textbf{Conclusione:} L'opera si conclude con una breve conclusione in cui Boccaccio parla dell'importanza dell'amicizia, dell'amore e della sorte.
\end{itemize}

\sdefinition{Metanarrazione}{
    La \textit{metanarrazione} è una tecnica narrativa,
    che consiste nell'intervento diretto dell'autore all'interno dello stesso testo che va componendo
    Si verifica così una narrazione che assume come proprio oggetto l'atto stesso del raccontare,
    così da sviluppare un romanzo nel romanzo.
}

\subsection{Analisi}

I primi 7 paragrafi [1-7] sono una scusa rivolte alle donne. Viene delineata la necessità di 
avere una salita di sofferenza prima del piacere.
Boccaccio spiega che l'introduzione (sofferente) è necessaria per capire come si arriva alle novelle. \\
La storia vera e propria comincia all'ottavo paragrafo [8] mediante l'arrivo della peste.
Questa peste, deriva geograficamente dall'Oriente prima di giungere a Firenze
(è stata trasportata dai topi sulle barche mercantili).
Secondo Boccaccio, è possibile che la pesta sia una punizione divina inflitta sugli uomini.
Viene descritto come la peste influenzasse il corpo, come lo facesse gonfiare e come i sintomi evolvessero per poi giungere alla morte.
Successicamente [13] si parla dello scopo dei medici; nessuna medicina riusciva a contrastare questa piaga, la quale è contagiosa [13]
anche fra uomini e animali [17].
Dopo aver descritto in dettaglio le dinamiche della malattia, vengono descritte
le reazioni delle persone [19]: 1. Forse il modo migliore è quello di vivere con parsimonia ma con cibi e vini di qualità
2. Fare festa e concedersi gli eccessi.
Le leggi divine e leggi umane sono finite, in quanto tutti sono affetti dal caos [23].
\\
Nei paragrafi [53-65] vi è l'inizio del lungo discorso di Pampinea, la quale arriva a proporre alle altre 6 conoscenti
di uscire da Firenze. Essa insiste molto che ciò non è una scelta di disimpegno,
al massimo sono loro ad essere il frutto dell'abbandono.
La proposta è di uscita, che tuttavia non è fuga nel divertimento, bensì di vita onesta nella campagna (contado),
onestà che non c'è più nella città. Lo scopo è quello di fare festa, provare piacere ed essere allegri,
ma \quotes{senza trapassare in alcun atto il segno della ragione}.
\sdefinition{Locus amoenus}{
    Un luogo naturale, verdeggiante, paradisiaco come edenico.
}
Nei seguenti paragrafi [-67] viene descritto un locus amoenus.
Queata descrizione è data come opposto al luogo di Firenze.
Questa non è una proposta di abbandona o fuga definitiva della città [71],
ma vedranno che cosa gli risparmia il cielo se rimarranno vivi.
Ma Filomena [74], la quale era molto discreta, ricorda che un gruppo di donne non può gestirsi
senza un uomo che le guidi, quanto le donne sono litigiose, sospettose, pusillanimi e paurosa.
Filomena le da ragione, ma Elissa fa notare che il progetto dell'onestà cadrebbe dall'inizio se
prendessero degli uomini non legati a loro (vivere sotto il medesimo tetto sarebbe uno scandalo).
Per pura coincidenza [78] 3 uomini entrano nella chiesa. Gli uomini sono chiamati Panfilo, Filostrato e Dioneo
e sono belli e di belle maniere, come lo sono le donne.
Questi 3 non sono completi estranei perché sono innamorati di alcune delle ragazze.
Neifile [81], l'amata di una dei ragazzi, parla bene dei ragazzi, ma teme
promiscuità siccome i ragazzi sono innamorati di loro e non sono i loro mariti.
La risposta a questa preoccupazione viene da Filomena, la quale dice che ciò non importa,
essendo il piano onesto. \\
Pampinea approccia i ragazzi con un sorriso e cominciano ad organizzare il trasferimento [88].
Usciti dalla città di misero in via. Il luogo scelto è solo a due miglia di distanza dalla città [89].
Nei successivi paragrafi viene descritto il luogo: un palazzo con sale e camere affrescate, cortile, loggie,
giardini meravigliosi con pozzi d'acqua freschissimi e cantine di vini raffinati.
Le prime parole sono di Dioneo, il quale dice si aver lasciato i propri pensieri luttuosi e tristi nella città.
Il suo programma consiste in una dicotomia: o le donne cantano e si divertono con lui
(sempre nei limiti dell'onestà), o lui se ne torna in città.
Pampinea propone di avere un campo per tenere la comunità entro i propri limiti e in armonia,
che si preoccupi di far star bene gli altri 9.
Affinché tutti abbiano ambo i ruoli, Pampinea indica una carica a rotazione giornaliera,
dove il primo capo è scelto collettivamente e i successivi vengono scelti dai predecessori.

\pagebreak

\subsection{Andreuccio da Perugia}

\subsubsection{Analisi}

Il turno di parola viene dato immediatamente alla fine della novella precedente,
il pezzo di cornice fra le due novelle è quindi assente. 

\textbf{Rubrica:} Andreuccio da Perugia, venuto a Napoli a comperar cavalli, in una notte
da tre gravi accidenti soprapreso, da tutti scampato con un rubino si torna a
casa sua.

% mercato, 500 fiorini d'oro, li mostra
% giovane (prostituta) e vecchia siciliana | la vecchia riconosce adreuccio e spiega alla giovane ocme sono imparentati
% la giovane torna a casa e mette e fa occupare la vecchia, mentre trama la trappola ad andreuccio
% Ella manda qualcuno a chiamare andreuccio, dicendo che una donna gli vuole parlare ed è apparecchiato.
% Il vanitoso andreuccio crede che lei sia innamorata di lui, e si reca in dimora.
% Questa donna siciliana lo aspetta in cima alla scala per abbracciarlo calorosamente.
% Piangendo di gioia, gli da il benvenuto e lo porta nella sua camera, la quale è stata preparata per essere ricca piuttosto che di una prostituta.
% Gli comunica di essere sua sorella spiegando che il padre, Pietro, avrebbe vissuto da giovane a Palermo,
% e che anreuccio sarebbe nato a Perugia da un'altra donna (sono fratellastri), per poi essere abbandonati dal padre.
% - re carlo
% Ella gli fa domande estensive circa i suoi parenti, rafforzando  la credibilità di tutta la questione

% Lo invita a cena, e si offre di avvisare l'albero di Andreuccio di avvisare della sua assenza per restare da lei.
% Lei fa durare tanto la cena, in maniera tale da far rimanere andreuccio a dormire
% Da una camera ad Andreuccio con un fanticello a sua disposizione.
% Si spoglia del farsello (tipico abito maschile) e dei pantaloni, appoggiandoli affianco al letto.
% chiede al fanciullo dove fosse il bagno, e lui gli mostra l'uscio.
% L'asse sul quale appoggia il piede è tagliato e casca nella latrina.
% La donna si affretta a controllare se avesse lasciato i suoi averi in camera.
% ANdreuccio si ritrova bloccato in strada, in mutande e imbrattato di feci.
% pronta a rientrare bussando violentemente contro il portone, ma viene ignorato e scacciato dai vicini.
% Giunge il rozzo e assonnato padrone della prostituta, che scaccia Andreuccio minacciandolo prima che possa finire di spiegarsi.

% Andreuccio perde la speranza di recuperare i suoi fiorini e cerca di tornare al suo albergo
% ma nella strada svolta per _Runa Catalana_ per andare a lavarsi.
% Vede due signori con degli arnesi, ma non riesce a nascondersi per il proprio tanfo.
% Andreuccio gli racconta tutta la storia, e i due capiscono da che casa provenisse.
% Tuttavia, i due lo rassicurano perché se non fosse caduto sarebbe morto nel sonno.
% I due gli propongono di andare con loro e ottenere più di quanto noa bbia perso.
% Sicome disperato, Andreuccio accetta prima di sapere di cosa si tratti.

% Il lavoro è il saccheggiamento della tomba di un vescovo morto recentemente
% il quale possiede un preziosissimo rubino.
% Prime di andare, si decidono di lavare Andreuccio in un pozzo.
% Giunti al pozzo, il secchio manca e quindi legano andreuccio e lo calano nel pozzo.
% Una volta lavato, la fune si rompe.
% Arrivano delle due guardie nel pozzo per bere, le quali tirano sù andreuccio pensando fosse il secchio
% Appena notano le mani di Andreuccio si spaventano e lasciano cadere la fune.
% Andreuccio incontra nuovamente i due ladri per strada e si dirigono alla chiesa per saccheggiare la tomba.
% Il coperchio è compsoto da una grande lastra di marmo.
% Una volta sollevata la lastra, i due vogliono far entrare Andreuccio, il quale si rende conto dei loro intenti.
% Alchè, una volta entrato, Andreuccio prende immediatamente l'anello e se lo nasconde
% prima di passare ai ladri tutti gli altri beni sul vescovo.
% I due ladri tolgono il supporto, facendo rimanere Andreuccio bloccato nella tomba.

% dopo un po' di tempo giungono sul posto altri ladri, i quali aprono la lastra.
% Andreuccio prende per le gambe il prete che stava entrando. I ladri fuggono lascindo la lastra aperta.
% Andreuccio se ne va con il rubino molto prezioso.

La vicenda può essere grossolanamente separata in tre avventure:
quella con la donna siciliana nel chiassetto (§§1-55), quella del pozzo (§§56-70)
ed infine quella della chiesa (§§71-89).
Le tre disavventure sono ambientate in luoghi distinti e stretti,
ma in tutti i posti vi è l'azione di cadere/scendere (latrina, pozzo, arca).
Ognuna di queste caduta rappresenta una crescita personale per Andreuccio, infatti,
ogni caduta è progressivamente più difficile da superare. Questo percorso rappresenta il suo
sviluppo personale, dove impara ad essere più scaltro e astuto.
Il sistema dei personaggi è costruito con Andreuccio al centro e tutti gli altri come antagonisti.
Andreuccio è molto ingenuo e privo di esperienza (§§3, §§16) ma anche vanitoso (§§11).
La prima azione da lui fatta mediante un ragionamento critico è quando si intasca l'anello per fregare i due ladri (§§77),
questo suo genio è caratterizato dai verbi \quotes{pensò} e \quotes{s'avisò}.
Prima di allora, Andreuccio aveva sempre agito in maniera completamente passiva o solo per disperazione (§§64).
\\
A differenza di Andreuccio, la prostituta è molto intelligente e possiede la capacità di agire in maniera critica e astuta.
La prima dimostrazione di ciò è quando nota la borsa di Andreuccio senza essere vista (§§4),
ma la dimostrazione della sua scaltrezza è data dalla sua recita dove convince Andreuccio di essere la sua sorellastra.
\\
La novella di Andreuccio ricorda ua fiaba ed un romanzo di formazione
Questo romanzo di formazione porta Andreuccio a crescere, acquisendo una maggiore astuzia e anche più fortuna di
quanta non ne avesse all'inizio, per cui Andreuccio acquista il pensiero di un mercante.
A lui non interessa diventare più saggio o colto, diventa più furbo e usa questa qualità per ottenere un guadagno materiale
Sia la prostituta che i ladri e Andreuccio guadagno qualcosa di materiale.
Tutti questi personaggi si arricchiscono in maniera immorale, senza nessun giudizio negativo dalla parte di Boccaccio.

Gli spazi della novella sono posti reali, la storia del protagonista comincia a Perugia,
si svolge a Napoli per poi tornare nuovamente a Perugia.
I posti descritti sono fattuali, come l'albergo e la casa siciliana, le vie di napoli citate
e il mercato dei cavalli.
Oltre ai posti, anche i riferimenti storici sono veri: Napoli non era infatti una città sicura.
Il vescovo e la sua morte sono reali, anche se di un'altro tempo. I personaggi presenti nella Novella
sono anch'essi probabilmente esistiti.

\subsubsection{Inferno, Canto XXIV}

\href{https://en.wikipedia.org/wiki/Vanni_Fucci}{Vanni Fucci}
incontra Dante nel canto XXIV. Fucci ha commesso il crimine di
rubare da una chiesa, per questo soffro in eterno nell'ottavo cerchio dell'Inferno.
Allo stesso modo, Andreuccio ruba in chiesa e se ne torna a casa con più soldi.
La differenza è che a Boccaccio non interessano le implicazioni morali delle azioni
delle persone, bensì considera solo la loro vita terrena.
Infatti, i personaggi che commettono peccati non vengono necessariamente visti
in maniera negativa.
Contrariamente, Dante ritiene che la vita sia solo un piccolo segmento di tempo
per prepararsi all'oltretomba, e si concentra proprio questo aspetto.

\pagebreak

\subsection{Lo stalliere del re Agilulfo}

\textbf{Rubrica:} Un pallafreniere giace con la moglie d'Agilulf re, di che Agilulf tacitamente s'accorge; truovalo e tondalo; il tonduto tutti gli altri tonde, e così scampa della mala ventura.

\sdefinition{Industria}{
    Con \textit{industria} si intende la capacità di chi acquista o recupera una cosa desiderata.
    Molto spesso la cosa desiderata è di tipo erotico e sessuale.
}

Le rr. 4-8 della novella dichiarano la morale ideologica del quale la novella sarà un esempio.
Questo concetto è quello che non sempre è auspicabile pretendere di sapere tutto,
perché a volte sapere certe cose porta allo svergognarsi e si ritorcono contro di noi.
La novella è quindi una dimosrazione di questo fenomeno.

La regina è vedova e sfortunata in amore (rr. 11),
mentre il re è bravo, saggio e capace governare.
Lo stalliere appartiene al ceto più basso, ma è intelligente e fisicamente simile al re.
La dinamica dei personaggi crea un triangolo amoroso, ma nonostante
i vertici del triangolo siano molto distanti da un punto di vista sociale, possono comunque battersi ad armi pari.
Gli ambiti dove si possono battere ad armi pari sono l'ingegno e la prestazione amorosa (rr. 58-59).
La donna cantanta è sempre posta ad un livello più alto e l'amore da parte dello stalliere
è di tipo platonico (rr. X).
Questo amore rimane segreto e arde come il fuoco (rr. 19), si innamora in maniera nobile e
cortese. Nonostante esso sia nobile d'animo, vuole anche avere delle effusioni carnali con la regina.

Il testo può essere principalmente diviso in due parti: nella prima lo stalliere
ha uno scopo erotico, e vuole andare a letto con la regina. Nella seconda,
vuole eludere la vendetta del re non facendosi scoprire.
In ambo le parti l'obiettivo è raggiunto mediante la furbizia, con una doppia beffa.
Per due volte vi è uno scambio di persona. Inizialmente, si scambia l'identità con il re, mentre
nella seconda diluisce la sua identità fra quelle dei suoi compagni.
Possiamo notare come tutti guadagnino qualcosa: la regina guadagna una notte più appagante,
il re guadagna reputazione con la regina e lo stalliere ci va a letto.

Lo stalliere fa finta di essere arrabiato per non parlare con la ragione, e
possiede un senso del limite, siccome si ferma prima per non rischiare di essere scoperto (rr. 54) (e non sfiderà più la fortuna).
Possiede la capacità di autocontrollo poiché sta fermo quando il re gli posa la mano
sul petto. Inoltre, compie la seconda beffa tagliando i capelli a tutti gli altri (rr. 95-96).
Le sue azioni rimarranno per sempre un segreto.
Questi sono i punti dove esso dimostra una grande furbizia e scaltrezza.
Similarmente, il re presente anche una grande intelligenza
perché riesce a contenersi quando scope che sua moglie è stata a letto con un altro,
riesce ad autocontrollarsi, quando un altro invece avrebbe fatto una scenata (rr. X).
Il re presume che il colpevole sia qualcuno di vicino (rr. 75) e ascolta i battiti
cardiaci degli stallieri. Anche quando scopre il colpevole, riesce a contenersi e gli taglia
i capelli per riconoscerlo il giorno successivo.
Il re riconosce inoltre il merito dell'avversario, dopo aver combattuto sullo stesso
piano nonostante essendo di ceti sociali completamente opposti (rr. 102-103).

Ci sono quindi degli ambiti, in questo caso, prestazione erotica e ignegno,
dove vengono completamente rimosse le condizioni sociali.

\pagebreak

\subsection{Lisabetta da Messina}

\textbf{Rubrica:} I fratelli d'Elisabetta uccidon
l'amante di lei: egli l'apparisce
in sogno e mostrale dove sia sotterato;
ella occultamente disotterra la testa e mettela in un testo di bassilico,
e quivi sù piagnendo ogni dì per una grande ora, i fratelli gliele tolgono,
e ella se ne muore di dolor poco appresso.

\snote{}{
    Nonostante ci siano 3 uomini, chi narra si rivolge spesso alle donne.
    Ciò è dato dalla lode di Boccaccio verso le donne.
}

\begin{enumerate}
    \item \textbf{Premessa (§3) e antefatto (§§4-5)}
    \item \textbf{Svolgimento dell'azione (§§6-23)}
    \begin{enumerate}
        \item Protagonisti: fratelli (§§6-11)
        \begin{itemize}
            \item Scoperta della trasca (§§6-11)
            \item Omicidio (§§8-9)
            \item Domande di Lisabetta (§§10-11)
        \end{itemize}
        \item Protagonista: Lisabetta (§§11-18)
        \begin{itemize}
            \item Sogno (§§12-13)
            \item Scoperta del cadavere e recupero della testa (§§14-16)
            \item Culto per il vaso (§§17-18)
        \end{itemize}
        \item Protagonisti: fratelli (§§19-22): sottrazione del vaso
        \item Protagonista: Lisabetta (§23): morte
    \end{enumerate}
    \item \textbf{Conclusione che rivela l'origine della novella (§§23-24)}
\end{enumerate}

% Struttura
% Sistema dei personaggi
% Interpretazioni

I protagonisti, essendo alternati, mostrano di non comunicare fra di loro.
Non vi sono dialoghi fra i fratelli e le sorelle vi sono solo domande che non vengono nemmeno risposte.
Piuttosto che essere verbali, i fratelli compiono azioni concrete. Sono uomini pratici.
Al contrario, la sorella è più verbale, comunica con le sue domande (§10, §11, §13, §20) e il pianto (§11, §12, §14, §17, §18, §20, §23).
L'unico punto in cui Elisabbeta sospende il pianto è quando trova il corpo deceduto e ne taglia la testa.

I personaggi sono divisi in due, da un lato i 3 fratelli mentre dall'altra Lisabetta, dall'altro Lisabetta, Fante e Lorenzo.
Il rapporto non è equilibrato, i fratelli sono gerarchicamente superiori per ragioni sociali.
I loro valori sono completamente diversi, mettono il denaro in cima a tutto. Tuttavia, ciò non viene detto direttamente,
bensì viene implicato dal loro interesse per la reputazione della loro famiglia ed attività.
Se gli interessi della sorella si dovessero scoprire, la loro reputazione, assieme ai loro affari, crollerebbero (§7, §22).

Quando uno dei fratelli scopre la sorella a letto con il garzone, non agisce di impulso ma ne parla con gli altri fratelli.
La vicenda viene trattata come se fosse un affare di famiglia; ragionano in maniera fredda e come mercanti, bilanciando cosa fare in funzione della resa economica.
La novella è tragica, nessuno ritrae un valore dalla vicenda.

Boccaccio accusa la logica mercantile (non il mondo dei mercanti) con i suoi eccessi.
Questa novella è in contrapposizione a quella dello Stalliere, dove viene posto un limite, mentre i fratelli non si fermano, non riuscendo ad equilibrare furbizia e sentimenti.

% Da un punto di vista psicoanaliti, a differenza di quello sociale, i ruoli si ivnertono:
% i 3 fratelli non hanno nome, indistinguibili e privi di identità.
% Sono 3 attori, ma solo un attante
% Non sanno amare, non sanno comunicare
% Di lorenzo si dice che tutti lor fatti guidava, per cui i fratelli sono anche incapaci senza di lui.
% Contrariamente, Lisabetta sa amare, sa comunicare (con le domande), e sa agire (prende l'iniziativa con Lorenzo, nonostante il divieto)
% È come se i fratelli vedessero la loro inferiorià nella sessualità della propria sorella.
% I fratelli uccidono Lorenzo 3vs1, ingannandolo e colpendolo alle spalle come dei vigliacchi.

% Quando Lisabetta sogna Lorenzo, le sue parole non sono nè di amore nè di enfasi verso ciò che è successo.
% Quando Lisabetta gli stacca la testa, la mette in grembo alla fante come se fosse un figlio.
% Lisabetta trasforma quindi il trauma portando tutte le sue cure alla testa, come se fosse un bambino.
% la pianta del basicili, mitologicamente, è legata alla fertilità. Ciô è legato alla cura e la crescita del figlio.
% Il basilico è bellissimo e odorifero molto §19 che è sintatticamente la stessa descrizione di Lorenzo, §5

\pagebreak

\subsection{Nastagio degli Onesti}

\textbf{Rubrica:} Nastagio degli Onesti, amando una de'Traversari,
spende le sue ricchezze senza essere amato.
Vassene, pregato da'suoi, a Chiassi; quivi vede cacciare ad un
cavaliere una giovane e ucciderla e divorarla da due cani.
Invita i parenti suoi e quella donna amata da lui ad un desinare,
la quale vede questa medesima giovane sbranare; e temendo di simile avvenimento
prende per marito Nastagio.

% 1 riga di cornice

% Personaggi 
I personaggi vengono descritti poco. Lei è \underline{molto} nobile (§4), e siccome sa di essere tale e bella
è molto cruda nei confronti di Nastagio (sdegnosa). Nastagio è nobile, giovane, molto ricco (dalla morte del padre),
spende senza misura e si toglierebbe la vita per quanto la ami. Il suo vizio di spendere senza ottenre nulla lo sta rovinando.

% I due racconti
La storia di Nastagio potrebbe essere quella di Guido, ed è identica rispetto alla sua.
Infatti, i loro nomi di famiglia coincidono. Entrambi sono gentiluomini di Ravenna
che amavano non essendo essi amati.
Le due donne non hanno nome, e il loro comportamento è simile (§6, §21).
La storia di Guido è tuttavia andata più avanti dal momento che lui si è effettivamente ucciso (§21), 
mentre l'altro ne ha solo avuto la tentazione (§6).
Inoltre, Nastagio si sforza di odiarla (§7) e Guido la odia e la uccide (§26).
Guido è quindi l'evoluzione possiible della storia di Nastagio.
La donna di Nastagia non lo odia, bensì ne è solamente indifferente (§6), mentre l'altra
non solo è indifferente, ma gode anche della sua morte (§22).

Nastagio vede la proiezione della sua storia in quella di Guido, e cambia il corso degli eventi per far sì che
la sua prenda un cammino diverso.
Dopo la sua visione, sfrutta la situazione di caccia che gli si presenta (§32) mediantre la propria furbizia.

Rovesciamento di quello che la religione indica come un obbliga, che qui viene espresso come un vizio: \\
La novella di Nastagio è una parodia di exemplum (storia di modello morale. E.g. fiaba, con fine didascalico).
Agire contro le forze di natura è sempre sbagliato. In questo caso, l'attrazione, è giusto che trovi sfogo.
Questa è una nuova morale insegnata da Boccaccio.
Il cambiamento inverosimile è che tutte le donne di Ravenna si concedono agli uomini (§44)
e X in una sera tramuta l'odio in amore.
È Dio che vuole che la crudeltà sia da condannare e che la pietà sia da lodare.
Tuttavia, viene successivamente indicata la crudeltà come il fatto di resistere alle forze della natura, per cui
vi è un rovescimaneto di ciò che insegna la religione.
Al contrario di Dante, finisce all'inferno chi resiste all'amore.

% La parodia
% L'industria di Nestagio
% L'aldilà
% Le fonti della novella

\snote{Rovesciamento Dante \(\rightarrow\) Boccaccio}{
    Dante scrive 100 canti nell'oltretomba e ne ambienta 2 sulla terra, mentre Boccaccio scrive 100 novelle di cui solamente
    due sono ambientate nell'oltretomba (tra l'altro, Inferno sulla terra).
    Il centro della vita non è più la preparazione all'oltretomba, bensì la vita stessa.
}

Nell'(anti-)exemplum di Passavanti la morale viene esplicitata alla fine: resistere è un valore,
mentre cedere è un disvalore.

% FONTI. Chiedi a qualcuno perché stavo studiando bio

\pagebreak

\subsection{Frate Cipolla}

\snote{}{
    Le novelle sono solitamente corte siccome la morale è centrata attorno una singola battuta.
    La novella di Frate Cipolla ne fa eccezione, compensando la brevità delle prime nove.
}

\textbf{Rubrica:} Frate Cipolla promette a certi contadini di mostrar loro la penna dell'agnolo Gabriello, in lugoo della quale trovando carboni, quegli dice esser di quegli che arostiron san Lorenzo.

Frate Cipolla è un frate di santo Antonio conosciuto per essere un bravo e scaltro oratore.
A messa terminata parla di una reliquia che avrebbe presentato, una piuma che fu una delle penne dell'Arcangelo Gabriele.
Due suoi amici nella chiesa si decidono di fargli una beffa e di sottrargliela.
Per eludere il fante ingaggiano una donna molto formosa per sedurlo, il quale viene distratto ma fallisce nel sedurre lei.
I due rubano la penna e la rimpiazzano con del barcone.
Il giorno dopo, Frate Cipolla si trova molti fedeli lì per vedere la piuma, ma quando ne apre la scatola
ci trova il carbone. Allora comincia a parlare del viaggio che ha compiuto per trovarla,
alla fine se la cava dicendo di aver preso la scatoletta per sbaglio contenente il carbone di San Lorenzo,
e quindi era un miracolo che Dio gli avesse fatto prendere il carbone proprio due giorni prima del giorno santo.

% Personaggi
% Gli elementi comici
%  - Il discordo di Guccio Imbrato
%  - Il discorso di Frate Cipolla
% Il pubblico 
% La polemica religiosa

La novella presente delle descrizioni eccezionalmente lunghe, in particolare addirittura tre.
Frate cipolla è piccolo, lievo nel viso, con capelli rossi (tratto da imbroglione) ma
soprattutto è bravo a parlare nonostante non abbia studiato (§7).
Guccio Imbratta, il fante di Frate Cipolla, viene presentato da Frate Cipolla,
mentre Frate Cipolla viene presentato dal narratore Dioneo.
Guccio viene presentato (§17) con nove aggettivi in rime da tre:
sugliardo e bugiardo; negligente, disubidente e maldicente; trascutato, smemorato e scostumato.
Il terzo personaggio (§21) è la serva Nuta, piccolo, brutta, sudata, unta, con un paio di poppe che sembravano due
cestoni di letame.
Le descrizioni di personaggi fatte da altri personaggi danno informazioni circa i loro gusti.

Il discoso di Guccio Imbratta, per cercare di conquistare la Nuta, è più corto del discorso di Frate Cipolla.
Questo discorso è come un'anticipazione del discorso che farà il amestro Frate.
La strategia di Frate Cipolla è un fiume di parole che in realtà hanno l'obiettivo di confondere il suo interlocutore.
Anche Guccio è abbastanza scaltro, ma lui fallisce nel sedurre la serva.
Frate cipolla dice di aver (§37) compiuto un viaggio dove sorge il sole (verso Oriente)
che in realtà è una frase ambigua perché il sole sorge quasi ovunque sul pianeta terra.
Oltre alla grande tecnica di spacciare per fantastiche delle cose che in realtà sono normalissime,
si inventa anche delle parole.

Il luogo della novella è la città di Boccaccio, Certaldo, per cui è come se l'autore prendesse in giro i suoi compaesani.
La novella è narrata da Dioneo.
All'interno della novella, Frate Cipolla racconta agli abitanti di Certaldo e ai suoi due amici (pubblico nascosto).
Il fatto che gli abitanti siano di Certaldo è rivolto agli altri nove novellatori.

Anche in questa novella il giudizio morale verso il protagonista è assente.
Nonostante Frate Cipolla sia un ingannatore, ne esce positivamente come scaltro.

\snote{Dante, Paradiso, XXIV, 124-126 (Beatrice)}{
    L'ordine di sant'Antonio era particolarmente avido e scaltro con gli imbrogli,
    come per esempio le indulgenze.
    % TODO metti i versi qua
}

\pagebreak

\subsection{Gianni Lotteringhi}

\textbf{Rubrica:} Gianni Lotteringhi ode di notte toccar l'uscio suo; desta la moglie, ed ella gli fa accredere che egli è la fantasima; vanno ad incantare con una orazione, ed il picchiare si rimane.

% Gianni ha successo nel suo lavoro ma è scarso altrove
% I frati di un convento nominano Capo Gianni (per sfruttarlo un po' perché ricco)
% Lui fornisce tessuti per i loro abiti in cambio di insegnamenti (preghiere)

% personaggi
La novella è caratterizzata da un triangolo amoroso fra Tessa, Gianni e Federigo.
Gianni ha avuto molta fortuna con il suo lavoro, ma altrove è piuttosto scarso.
Esso è una persona molto ingenua, e per quanto riguarda la religione è molto
superstizioso e bigotto (§§4-5).
Federigo è bello, giovane e fresco (§6).
Rispetto a Gianni, Federgio è più intelligente e allegro.
I due sono infatti molto distanti.
% differenze
Il marito Gianni si mangia la carna salata, mentre l'amante si gode la cena completa (§§12-13).
La moglie con il marito dorme e basta, mentre con l'amante si sfoga in effusioni sessuali (§8 e §20).
Inoltre, un altro elemento di differenza è quello delle preghiere. Infatti, Gianni intende le preghiere in maniera letteraria,
mentre Federigo riesce ad interderne il significato (§8 e §20).
%
Tessa è bella, intelligente e saggia (§6). Rispetto è marito è molto più astuta, ma è innamorata di Federigo.
Conosce bene il marito e sa di poterlo inganare (§7).
Il marito viene ingannato dicendogli che la preghiera può ora essere detto dal momento che sono
entrambi presenti, facendolo sentire importante.


% non è importante che ci siano "2 versioni"
% Il consiglio di Emilia alle altre novellatrici è letterale; la beffa è utile per il tradimento

\pagebreak

\part{Petrarca (1304 - 1374)}

%\section{Biografia}

%TODO

\section{Rerum vulgarium fragmenta (Il Canzoniere)}

Il testo è composto da 366 poemi e parla del suo amore tormentato per una donna di nome Laura.
Tuttavia, il protagonista del libro non è Laura, bensì Petrarca che ne parla.
% L'opera è separata in 2.

\subsection{Voi ch'ascoltate in rime sparse il suono}

La prima poesia è un sonetto con schema delle rime ABBA, ABBA, CDE, CDE.

\begin{center}
    \textit{\textbf{Voi ch'ascoltate} in \textbf{rime sparse} il suono} \\
    \textit{di quei sospiri ond'io nudriva 'l core} \\
    \textit{in sul mio primo giovenile errore} \\
    \textit{quand'era in parte altr'uom da quel ch'i' sono,}
\end{center}
\begin{center}
    \textit{del vario stile in ch'io piango et ragiono} \\
    \textit{fra le vane speranze e 'l van dolore,} \\
    \textit{ove sia chi per prova intenda amore,} \\
    \textit{\textbf{spero trovar} pietà, nonché perdono.}
\end{center}
\begin{center}
    \textit{Ma ben veggio or sì come al popol tutto} \\
    \textit{favola fui gran tempo, onde sovente} \\
    \textit{di me medesmo meco mi vergogno;}
\end{center}
\begin{center}
    \textit{et del mio vaneggiar vergogna è 'l frutto,} \\
    \textit{e 'l pentersi, e 'l conoscer chiaramente} \\
    \textit{che quanto piace al mondo è breve sogno.}
\end{center}

Questo primo sonetto fa da prologo, ma possiede anche degli elementi di bilancio,
cioè degli elementi che leggono l'esperienza del libro retroaspettivamente,
e quindi fa anche da epilogo.
Inizialmente l'autore si rivolge al lettore che ascolta (v. 1), come spesso di tradizione (es. Il Paradiso II).
le prime due quartine presentano un errore di sintassi; nonostante si rivolga inizialmente all'autore,
il verbo principale dopo le varie subordinate è \quotes{spero} (v. 8), causando un incoerenza fra soggetto e verbo.
Questo è un elemento che fa da prologo, ossia la forma del suo libro è data da \quotes{rime sparse} (v. 1).
Nonostante ciò, vi è un significato che percorre le poesie nel loro ordine.
Dopo aver annunciato la forma del libro, viene anche annunciato il tema, ossia
un tema di una natura amorosa (v. 2).
Il (v. 3) fa invece da epilogo, perché l'amore, ormai passato, viene definito come errore.
L'\textit{errore} per questa donna viene giudicato tale perché esso risiede nell'atto stesso di amarla,
ossia il fatto che si tratti di un amore nei confronti di una donna terrena, mentre l'unico amore
vero può essere rivolto solo verso Dio. L'errore è appunto quello di essersi dedicati a qualcosa di fugace,
uno sviamento rispetto alla via verso Dio ed il suo amore eterno.
Tuttavia, questo errore non è ancora completamente superato poiché Petrarca
era \textit{in parte} un uomo diverso da ciò che è oggi (v. 4), e quindi il cambiamento non è ancora
completamente compiuto.
In fondo, possiamo notare un conflitto intrinseco fra l'amore sacro e l'amore terreno.
Questo sonetto è stato quindi scritto dopo tutto il resto, e potrebbe anche essere posizionato
logicamente al termine.
\\
Nella seconda quartina il poeta dichiarare di sperare di trovare pietà e perdono
presso un pubblico più ristretto rispetto a quello della prima quartina,
ossia presso coloro che nell'amore hanno fatto esperienza (v. 7).
Il \textit{vario stile} (v .5) si riferisce all'intercambiarsi continuo fra
razionalità e luciditià e pianto e dolore. Infatti, vi è chiasmo con le parole \textit{piango},
\textit{ragion}, \textit{speranze} e \textit{dolore}, dove all'interno vi è la razionalità e all'esterno
parole negative di dolore.
L'errore è quello di essersi attaccati ad un qualcosa di \textit{vano}, ma non inutile, bensì effimera, e quindi
indegna di un amore che dovrebbe essere dedicata solo alle cose Celesti.

\pagebreak

\part{Analisi dei testi}

% versi tronchi piani e sfruccioli | accenti tonici etc.
% le rime si considerano tali dalle medesime lettere dopo l'ulima vocale con accento tonico

L'analisi di un testo viene separata nell'\textit{analisi metrica} e nell'\textit{analisi}.

\paragraph{Analisi metrica}

\phantom{ }\vspace{0.1cm}
\sexample{Analisi metrica}{
   Sonetto con schema ABBA, ABBA, CDE, CDE.
}

\paragraph{Analisi}

Perifrasi = giro di parole

\section{Analisi delle novelle}

\begin{enumerate}
    \item Struttura
    \item Personaggi
    \item Tempo e spazio
    \item Novella e realtà storica
    \item Interpretauione storica-ideologica
\end{enumerate}

\end{document}
