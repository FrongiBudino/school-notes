\documentclass[a4paper]{article}

\usepackage{amsmath}
\usepackage{amssymb}
\usepackage{parskip}
\usepackage{fullpage}
\usepackage{hyperref}
\usepackage{stellar}
\usepackage{soul}

\hypersetup{
    colorlinks=true,
    linkcolor=black,
    urlcolor=blue,
    pdftitle={Italiano},
    pdfpagemode=FullScreen,
}

\title{Italiano}
\author{Paolo Bettelini}
\date{}

\newcommand{\quotes}[1]{``#1''}

\newcommand\hr{\par\vspace{-.5\ht\strutbox}\noindent\hrulefill\par\vspace{0.15cm}}

\begin{document}

\maketitle
\tableofcontents
\pagebreak

% Autori:
% Dante, Boccaccio, Petrarca
% Machiavelli (sagiistica politica)
% (sagiistica politica) e Ariosto
% Beccaria
% Leopardi
% Leopardi

% Esame scritto: 2 traccie nuove e sceglierne una
% commentare 4 ore di orologio

% Porta un dizionario
% Sufficientemente grande da avere le parole alma guiderdone spirto

% Verifiche:
% 11 ott, Dante 
% 6 dic, Boccaccio(?)



\part{Dante (1265-1321)}

\section{Biografia}

La biografia di Dante è molto offuscata e nessuno scritto originale è rimasto.
Questi fattori rendono difficile datare le sue varie opere e contestualizzarle. Inoltre,
è anche difficile validare in maniera precisa le parole esatte scritte dall'autore, siccome i testi che possediamo
sono frutto di trascrizioni.

Dante nasce a Firenze nel 1265 in una piccola nobilità cittadina.
A 12 anni diventa promesso sposo di a Gemma Donati.

A 18 anni incontra Beatrice, dopo averla vista per la prima volta a 9 anni.

Verso l'anno 1295 Dante, si avvicina alla politica.
Si iscrive all'Arte (Arte dei medici e degli speziali), questo è dato dal fatto che essere iscritti ad un Arte
fosse un requisito necessario per esercitare un'attività politica. Dante diventa piore, per cui a capo della cittadina. %%

Prima della nascita di Dante, i Ghibellini sostenevano il potere dell'imperatore, mentre i Guelfi sostenevano quello papa.
Le due parti erano in forte conflitto, e nella battaglia del 1266, muore il figlio dell'imperatore.
I Ghibellini escono quindi di scena quando Dante è appena nato.

Successivamente, i Guelfi si separano in Bianchi e Neri, con un conflitto ancora più forte di quello precedente.

La scena politica fiorentina era dominata dallo scontro fra i Bianchi e i Neri.
Dante, durante il suo priorato, manda in esilio in più violenti dei Neri, fino alla scaduta del suo priorato.
Papa Bonificio VIII manda le truppe di Carlo di Valois, le quali permettono ai Neri di prendere carica al governo.
I bianchi vengono quindi esiliati, fra cui Dante. 

% Boccaccio - primo grande studiso di dante

\section{Vita Nova}

\subsection{Introduzione}

\sdefinition{Sonetto}{
    Il \textit{sonetto} è una poesie composto da 2 quartine e 2 terzine dove tutti i versi sono degli endecasillabi.
}

\sdefinition{Prosimetro}{
    Il \textit{prosimetro} è un testo ibrido, composto da un
    racconto (prosa) intervallato e poesie (versi).
}

La \textit{Vita nova} è il primo prosimetro. Esso racconta la storia d'amore da parte di Dante
nei confronti di Beatrice.
Questa vicenda diventa un modello per questa tipologia di narrativa.

Il significato del titolo indica come Dante consideri l'inizio della sua vita (rinnovata)
quando vide Beatrice per la prima volta.
Il primo contatto amoroso nella poesie è sempre caratterizzato
dall'innamoramento a prima vista.

% Il saluto nel medioevo da parte delle donne
Una volta che la voce dell'interesse di Dante verso Beatrice le giunge, lei gli nega il saluto.
Tuttavia, Dante continua ad esprimere il suo amore verso Beatrice semplicemente
lodandola (scrivendo di lei), completamente senza ricambio di interesse.

Questo libro introduce la simbologia del numero 9 associato a Beatrice.
Questo è dato dal fatto che l'abbia vista per la prima volta a 9 anni, rivista 9 anni dopo,
e altri motivi che vengono descritti. Il numero 9 è anche un simbolo biblico (3 volte la trinità).

\subsection{Oltre la spera}

\begin{center}
    \textit{Oltre la spera che più larga gira,} \\
    \textit{passa 'l sospiro ch'esce del mio core:} \\
    \textit{intelligenza nova, che l'Amore} \\
    \textit{piangendo mette in lui, pur sù lo tira.}
\end{center}

Il primo verso è una perifrasi che indica "oltre il pianeta più lontano", ossia il paradiso
siccome la visione dell'universo era quella tolemaica.

Il secondo verso ci dice che il sospiro del poeta esce dal suo cuore, mentre è vivo, dalla sua intimità più profonda,
e attraverso i cieli fino al paradiso.  

I versi 3-4 descrivono ciò che permette questo percorso, ossia ciò che lo tira
verso l'alto. Questa forza è un'intelligenza nova, ossia una nuova sensibilità nel vedere le cose.
Questa nuova intelligenza deriva dall'amore, che permette all'autore di avere una nuova consapevolezza.
Questa esperienza amorosa è dolorosa ma porta ad una nuova capacità di intendimento.
Inoltre, la parola amore ha la maiuscola perché esso viene personificato.

\begin{center}
    \textit{Quand'elli è giunto là dove disira,} \\
    \textit{\textbf{vede} una donna, che riceve onore,} \\
    \textit{e \textbf{luce} sì, che per lo suo \textbf{splendore}} \\
    \textit{lo peregrino spirito la \textbf{mira}.}
\end{center}

La seconda quartina è il punto di arrivo.
Quando lo spirito arriva, vede una donna.
Essa viene onorata dagli altri beati, Dio e la madonna.
Viene anche detto che questa donna brilla.
A causa di questo splendore, lo spirito giunto in paradiso (pellegrino, in pellegrinaggio) la ammira.

\begin{center}
    \textit{\textbf{Vedela} tal, che quando 'l mi ridice,} \\
    \textit{io no lo \textbf{'ntendo}, sì \textbf{parla} sottile} \\
    \textit{al cor dolente, che lo fa \textbf{parlare}.}
\end{center}

Lo spirito ripercorre il medesimo tragitto verticale, ma al contrario, tornando da Dante.
Questo spirito cerca di spiegargli che cosa ha visto.
"La vede tale che quando me lo ridice, io non capisco".
Dante non comprende quindi ciò che lo spirito gli riferisce, perché
"parla sottile", ossia parla in maniera troppo difficile.
Il cuore dolente del poeta è ciò lo fa sì che lo spirito venga interrogato.
Infatti, lo spirito parla proprio al cuore \underline{e} a Dante (questo amplifica l'incomprensione della spiegazione).
Lo spirito parla in maniera troppo complessa perché il linguaggio non riesce
ad esprimere quello che si è provato (topos dell'ineffabilità, è ineffabile)
siccome l'esperienza lo tracende.

\begin{center}
    \textit{So io che \textbf{parla} di quella gentile,} \\
    \textit{però che spesso \textbf{ricorda} Beatrice,} \\
    \textit{\st{sì ch'io lo \textbf{'ntendo} ben, donne mie care.}}
\end{center}

\textbf{\color{red}nota:} La parola "però" significa "per ciò". \\
Questa è l'unica occorrenza dove Beatrice viene nominata direttamente in un testo poetico in \textit{Vita Nova}.\\
Nell'incomprensione fra Dante e lo spirito, Dante capisce che la donna vista era sicuramente
Beatrice. \\
Nella poesia antica, la parola \textit{gentile} è molto più profonda di quella odierna
e possiede un significato diverso. Essa ha un significato nobile di purezza (nobiltà d'animo).

L'ultimo verso è dato dal fatto che Dante si stesse riferendo a delle Donne.

\hr

Questo sonetto è diviso in due parti, dove vengono distinte le due verticalità del viaggio dello spirito
(avanti e indietro).

Molte parole della prima parte appartendono alla sfera visiva, poiché il paradiso
è fatto di luci, mentre molte parole della seconda fanno parte del parlare.
Questo è dato dal fatto che lo spirito può vedere, ma ha l'impossibilità di esprimersi.

Questa separazione è collegata dall'uso di due parole quasi uguali,
\textbf{mira} e \textbf{Vedela} (detto per anadiplosi).

\section{La Divina Commedia}

\pagebreak

\part{Analisi dei testi}

% versi tronchi piani e sfruccioli | accenti tonici etc.
% le rime si considerano tali dalle medesime lettere dopo l'ulima vocale con accento tonico

L'analisi di un testo viene separata nell'\textit{analisi metrica} e nell'\textit{analisi}.

\paragraph{Analisi metrica}

\phantom{ }\vspace{0.1cm}
\sexample{Analisi metric}{
   Sonetto con schema ABBA, ABBA, CDE, CDE.
}


\paragraph{Analisi}

% perifrasi = giro di parole

\end{document}
