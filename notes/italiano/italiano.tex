\documentclass[a4paper]{article}

\usepackage{amsmath}
\usepackage{amssymb}
\usepackage{parskip}
\usepackage{fullpage}
\usepackage{hyperref}
\usepackage{stellar}
\usepackage{soul}
\usepackage{tikz}
\usepackage{graphicx}

\usetikzlibrary{cd}

\hypersetup{
    colorlinks=true,
    linkcolor=black,
    urlcolor=blue,
    pdftitle={Italiano},
    pdfpagemode=FullScreen,
}

\title{Italiano}
\author{Paolo Bettelini}
\date{}

\newcommand{\quotes}[1]{``#1''}

\newcommand\hr{\par\vspace{-.5\ht\strutbox}\noindent\hrulefill\par\vspace{0.15cm}}

\begin{document}

\maketitle
\tableofcontents
\pagebreak

% Autori:
% Dante, Boccaccio, Petrarca
% Machiavelli (sagiistica politica)
% (sagiistica politica) e Ariosto
% Beccaria
% Leopardi
% Leopardi

% Esame scritto: 2 traccie nuove e sceglierne una
% commentare 4 ore di orologio

% Porta un dizionario
% Sufficientemente grande da avere le parole alma guiderdone spirto

% Verifiche:
% 11 ott, Dante 
% 6 dic, Boccaccio(?)

% TODO, spostare definizioni in un punto apposta



\part{Dante (1265-1321)}

\section{Biografia}

La biografia di Dante è molto offuscata e nessuno scritto originale è rimasto.
Questi fattori rendono difficile datare le sue varie opere e contestualizzarle. Inoltre,
è anche difficile validare in maniera precisa le parole esatte scritte dall'autore, siccome i testi che possediamo
sono frutto di trascrizioni.

Dante nasce a Firenze nel 1265 in una piccola nobilità cittadina.
A 12 anni diventa promesso sposo di a Gemma Donati.

A 18 anni incontra Beatrice, dopo averla vista per la prima volta a 9 anni.

Verso l'anno 1295 Dante, si avvicina alla politica.
Si iscrive all'Arte (Arte dei medici e degli speziali), questo è dato dal fatto che essere iscritti ad un Arte
fosse un requisito necessario per esercitare un'attività politica. Dante diventa piore, per cui a capo della cittadina. %%

Prima della nascita di Dante, i Ghibellini sostenevano il potere dell'imperatore, mentre i Guelfi sostenevano quello papa.
Le due parti erano in forte conflitto, e nella battaglia del 1266, muore il figlio dell'imperatore.
I Ghibellini escono quindi di scena quando Dante è appena nato.

Successivamente, i Guelfi si separano in Bianchi e Neri, con un conflitto ancora più forte di quello precedente.

La scena politica fiorentina era dominata dallo scontro fra i Bianchi e i Neri.
Dante, durante il suo priorato, manda in esilio in più violenti dei Neri, fino alla scaduta del suo priorato.
Papa Bonificio VIII manda le truppe di Carlo di Valois, le quali permettono ai Neri di prendere carica al governo.
I bianchi vengono quindi esiliati, fra cui Dante. 

% Boccaccio - primo grande studiso di dante

\section{Vita Nova}

\subsection{Introduzione}

\sdefinition{Sonetto}{
    Il \textit{sonetto} è una poesie composto da 2 quartine e 2 terzine dove tutti i versi sono degli endecasillabi.
}

\sdefinition{Prosimetro}{
    Il \textit{prosimetro} è un testo ibrido, composto da un
    racconto (prosa) intervallato e poesie (versi).
}

La \textit{Vita nova} è il primo prosimetro. Esso racconta la storia d'amore da parte di Dante
nei confronti di Beatrice.
Questa vicenda diventa un modello per questa tipologia di narrativa.

Il significato del titolo indica come Dante consideri l'inizio della sua vita (rinnovata)
quando vide Beatrice per la prima volta.
Il primo contatto amoroso nella poesie è sempre caratterizzato
dall'innamoramento a prima vista.

% Il saluto nel medioevo da parte delle donne
Una volta che la voce dell'interesse di Dante verso Beatrice le giunge, lei gli nega il saluto.
Tuttavia, Dante continua ad esprimere il suo amore verso Beatrice semplicemente
lodandola (scrivendo di lei), completamente senza ricambio di interesse.

Questo libro introduce la simbologia del numero 9 associato a Beatrice.
Questo è dato dal fatto che l'abbia vista per la prima volta a 9 anni, rivista 9 anni dopo,
e altri motivi che vengono descritti. Il numero 9 è anche un simbolo biblico (3 volte la trinità).

\subsection{Oltre la spera}

\begin{center}
    \textit{Oltre la spera che più larga gira,} \\
    \textit{passa 'l sospiro ch'esce del mio core:} \\
    \textit{intelligenza nova, che l'Amore} \\
    \textit{piangendo mette in lui, pur sù lo tira.}
\end{center}

Il primo verso è una perifrasi che indica "oltre il pianeta più lontano", ossia il paradiso
siccome la visione dell'universo era quella tolemaica.

Il secondo verso ci dice che il sospiro del poeta esce dal suo cuore, mentre è vivo, dalla sua intimità più profonda,
e attraverso i cieli fino al paradiso.  

I versi 3-4 descrivono ciò che permette questo percorso, ossia ciò che lo tira
verso l'alto. Questa forza è un'intelligenza nova, ossia una nuova sensibilità nel vedere le cose.
Questa nuova intelligenza deriva dall'amore, che permette all'autore di avere una nuova consapevolezza.
Questa esperienza amorosa è dolorosa ma porta ad una nuova capacità di intendimento.
Inoltre, la parola amore ha la maiuscola perché esso viene personificato.

\begin{center}
    \textit{Quand'elli è giunto là dove disira,} \\
    \textit{\textbf{vede} una donna, che riceve onore,} \\
    \textit{e \textbf{luce} sì, che per lo suo \textbf{splendore}} \\
    \textit{lo peregrino spirito la \textbf{mira}.}
\end{center}

La seconda quartina è il punto di arrivo.
Quando lo spirito arriva, vede una donna.
Essa viene onorata dagli altri beati, Dio e la madonna.
Viene anche detto che questa donna brilla.
A causa di questo splendore, lo spirito giunto in paradiso (pellegrino, in pellegrinaggio) la ammira.

\begin{center}
    \textit{\textbf{Vedela} tal, che quando 'l mi ridice,} \\
    \textit{io no lo \textbf{'ntendo}, sì \textbf{parla} sottile} \\
    \textit{al cor dolente, che lo fa \textbf{parlare}.}
\end{center}

Lo spirito ripercorre il medesimo tragitto verticale, ma al contrario, tornando da Dante.
Questo spirito cerca di spiegargli che cosa ha visto.
"La vede tale che quando me lo ridice, io non capisco".
Dante non comprende quindi ciò che lo spirito gli riferisce, perché
"parla sottile", ossia parla in maniera troppo difficile.
Il cuore dolente del poeta è ciò lo fa sì che lo spirito venga interrogato.
Infatti, lo spirito parla proprio al cuore \underline{e} a Dante (questo amplifica l'incomprensione della spiegazione).
Lo spirito parla in maniera troppo complessa perché il linguaggio non riesce
ad esprimere quello che si è provato (topos dell'ineffabilità, è ineffabile)
siccome l'esperienza lo tracende.

\begin{center}
    \textit{So io che \textbf{parla} di quella gentile,} \\
    \textit{però che spesso \textbf{ricorda} Beatrice,} \\
    \textit{\st{sì ch'io lo \textbf{'ntendo} ben, donne mie care.}}
\end{center}

\textbf{\color{red}nota:} La parola "però" significa "per ciò". \\
Questa è l'unica occorrenza dove Beatrice viene nominata direttamente in un testo poetico in \textit{Vita Nova}.\\
Nell'incomprensione fra Dante e lo spirito, Dante capisce che la donna vista era sicuramente
Beatrice. \\
Nella poesia antica, la parola \textit{gentile} è molto più profonda di quella odierna
e possiede un significato diverso. Essa ha un significato nobile di purezza (nobiltà d'animo).

L'ultimo verso è dato dal fatto che Dante si stesse riferendo a delle Donne.

\hr

Questo sonetto è diviso in due parti, dove vengono distinte le due verticalità del viaggio dello spirito
(avanti e indietro).

Molte parole della prima parte appartendono alla sfera visiva, poiché il paradiso
è fatto di luci, mentre molte parole della seconda fanno parte del parlare.
Questo è dato dal fatto che lo spirito può vedere, ma ha l'impossibilità di esprimersi.

Questa separazione è collegata dall'uso di due parole quasi uguali,
\textbf{mira} e \textbf{Vedela} (detto per anadiplosi).

% TODO atoni e tonici

\sdefinition{Legge di Tobler Mussafia}{
    È vietato iniziare un verso (poesie o prosa) o far seguire una congiunzione coordinante
    con un pronome atoni.
}

\section{La Divina Commedia}

\subsection{Cosmo Dantesco}

\sdefinition{Sistema tolemaico}{
    Data la credenza creazionista, Dio ha creato l'uomo e l'ha collocato al centro.
    Per cui, la Terra risiede al centro del sistema solare, dove gli altri pianeti gli ruotano attorno.
}

Le colonne d'Ercole (Stretto di Gibilterra) e La foce del Gange
sono i due estremi della Terra. All'uomo non è concesso conoscere oltre questi confini
(fare ciò implicherebbe peccare di superbia).

Gerusalemme si trova al centro dell'emisfero. Sotto di esso, risiede l'inferno.

Lucifero era l'angelo prediletto di Dio.
Lucifero si ribella a Dio, e per punizione viene scagliato sulla Terra, la quale,
prova ribrezzo e si ritira formando la forma conica dell'inferno. Lucifero si trova nel punto
più profondo dell'inferno, ossia il centro della Terra, nonché il punto più lontano da Dio.

La creazione dell'inferno crea una montagna dall'altra parte del mondo, dove in cima ad esso
vi è il Giardino dell'Eden. Ciò marca anche la creazione del purgatorio.

\subsection{L'inferno}

L'inferno è composto da settori sempre più stretti. Più lo spazio diminuisce e più i peccati sono immorali
secondo Date.
Principalmente, l'inferno è suddiviso in 3 sezioni.
Dall'alto verso il basso, ci sono gli \textit{incontinenti}, \textit{violenti} e
i \textit{freudolenti}.

\sdefinition{Legge del contrappasso}{
    La \textit{legge del contrappasso} associa una pena
    legata alla colpa.
    Il nesso avviene o per \textit{analogia} o per \textit{opposizione}.
}

%\subsection{Struttura dell'inferno}
%\begin{figure}[h]
%    \centering
%    \includegraphics[width=0.5\textwidth]{./inferno.jpg}
%\end{figure}

\subsection{Struttura del testo}

% https://tikzcd.yichuanshen.de/#N4Igdg9gJgpgziAXAbVABwnAlgFyxMJZABgBoBGAXVJADcBDAGwFcYkQAdDnGADx2ABjemDyCAFjAC+IKaXSZc+QinIVqdJq3Zce-XMAAKzAE4Bzejggn8MuQux4CRNcQ0MWbRJ2588AgEkwADMYE0g7eRAMR2UXUgAmdy0vHz1-I3oTeigsOAhIhyVnFATSNxoPbW9dP2ByAGoAZiaAAmFRLELoxScVZDKqSpSdX35gFvaRPG6Y4v6ypOHPUfSJto6Z2Q0YKDN4IlBgkwgAWyQykCskABYaSRz2HAB3CAeoBHsQY7OkNSuIEgAKz3GCPbwvN5gj6yKI-c6IJo0a6IABsoPBV1e70+cJOCLIAL+X3hSEJKMuACMYGAoEgmsQSfiycjAYiaNTaUgALQMqSUKRAA
\begin{center}
\begin{tikzcd}
    & \textit{Inferno} \arrow[r, two heads]    & \text{1+33 canti} \\
    \text{Cantiche} \arrow[r] \arrow[ru, bend left] \arrow[rd, bend right] & \textit{Purgatorio} \arrow[r, two heads] & \text{33 canti}   \\
    & \textit{Paradiso} \arrow[r, two heads]   & \text{33 canti}  
\end{tikzcd}
\end{center}

\subsection{Lucifero}

Lucifero viene rappresentato come un grande orrenda cretura con 3 bocche.
In ogni bocca mastica per l'eternità i 3 peccatori più grandi.
Al centro Giuda, mentre ai lati Bruto e Cassio.

\pagebreak

\subsection{Inferno, I, 1-12}

\begin{center}
    \textit{Nel mezzo del cammin di nostra vita} \\
    \textit{mi ritrovai per una selva oscura,} \\
    \textit{ché la diritta via era smarrita.}
\end{center}

la vita media è di 70 anni. Questo è un dato biblico e non il valore della vita media.
Dante è nato nel 1265, per cui 1265+35=1300. Il percorso inizia infatti esattamente nel 1300.
Questo dato viene anche confermato in \textit{Inferno XXI, 112-114}.

\sdefinition{Giubileo}{
    Il \textit{giubileo} è un anno di assoluzione collettica di peccati.
}
Questa data è quella del primo Giubileo, indetto dal Papa Bonifacio VIII.

\sdefinition{Allegoria}{
    Figura retorica per mezzo della quale l'autore esprime e il lettore ravvisa un significato riposto,
    diverso da quello letterale.
}
La selva oscura, dove non vi è luce, rappresenta una condizione di peccato.
La \textit{diritta via} è quella che conduce a Dio.
Essa è smarrita, ma può essere appunto ritrovata.
\\
Dante non specifica il tipo di peccato, questo è dato dal fatto che Dante rappresenta allegoricamente
l'interno dell'umanità (nel 1300), per cui il peccato di tutti gli uomini in quel periodo.

\begin{center}
    \textit{Ahi quanto a dir qual era è cosa dura} \\
    \textit{esta selva \textbf{selvaggia} e \textbf{aspra} e \textbf{forte}} \\
    \textit{che nel pensier rinova la paura!}
\end{center}

I tre aggettivi; selvaggia (disumano), aspra (fitta) e forte (da cui è difficile uscire)
sono disposti a climax.

\begin{center}
    \textit{Tant' è amara che poco è più morte;} \\
    \textit{ma per trattar del ben ch'i' vi trovai,} \\
    \textit{dirò de l'altre cose ch'i' v'ho scorte.}
\end{center}

La morte, che è la cosa più terribile che ci sia, lo è solamente poco più della selva.
\\
I due verbi sui quali si chiude \textit{Vita Nova},
\textbf{dire} e \textbf{trattare}, si ritrovano all'inizio della \textit{Commedia}.
In \textit{Vita Nova} questi verbi si riferiscono all'io poetico, mentre all'inizio della \textit{Commedia}
sono riferiti al \textbf{bene} e ad \textbf{altre cose}.
Il bene si riferisce alla salvezza (Dio), mentre altre cose si riferisce a tutto ciò che trovò durante il viaggio.
La parola \textit{vi} si riferisce probabilmente a tutto il viaggio compiuto da Dante.

\begin{center}
    \textit{Io non so ben ridir com' i' v'intrai,} \\
    \textit{tant' era pien di sonno a quel punto} \\
    \textit{che la verace via abbandonai.}
\end{center}

Il sonno rappresenta in senso allegorico il sonno della coscienza, che porta al peccato, ossia la selva.

\snote{Funzioni di Dante}{
    Date ha diverse funzioni che si intrecciano nel racconto.
    \begin{itemize}
        \item Dante personaggio, pellegrino che compie il viaggio
        \item Dante allegoria per tutta l'umanità
        \item Dante poeta fiorentino
    \end{itemize}
}

\begin{center}
    \textit{Ma poi ch'i' fui al piè d'un colle giunto,} \\
    \textit{là dove terminava quella valle} \\
    \textit{che m'avea di paura il cor compunto,}
\end{center}

\begin{center}
    \textit{guardai in alto e vidi le sue spalle} \\
    \textit{vestite già de' raggi del pianeta} \\
    \textit{che mena dritto altrui per ogne calle.}
\end{center}

Le spalle del colle sono il punto in cui la collina si piega.
Questa perifrasi indica semplicemente che la collina era illuminata dalla luce solare.
In alto vi è la luce divina, mentre in basso c'è il buio del peccato.
Il collo rappresenta infatti il percorso difficile; è molto più facile
cadere all'inferno che giungere a Dio. \\
Il gesto di guardare in alto indica un progressivo distaccarsi dal peccato.

\begin{center}
    \textit{Allor fu la paura un poco queta,} \\
    \textit{che nel lago del cor m'era durata} \\
    \textit{la notte ch'i' passai con tanta pieta.}
\end{center}

\begin{center}
    \textit{E come quei che con \textbf{lena affannata},} \\
    \textit{uscito fuor del pelago a la riva,} \\
    \textit{si volge a l'acqua perigliosa e guata,}
\end{center}

% pelago = mare
Il verbo \textbf{guatare} significa guardare con partecipazione, spesso paura.
Questa similitudine mette in confronto un naufrago che scampa il pericolo dell'acqua,
che come Dante scampa dalla selva e si gira a guardarla, con un sentimento di sollievo.
Il corpo di Dante è fermo, ma il suo animo è ancora spaventato e vorrebbe continuare a scappare.

Questo indica anche che indugiare nel peccato, come indugiare nella selva o nelle acque, porta alla morte.

\begin{center}
    \textit{così l'animo mio, ch'ancor fuggiva,} \\
    \textit{si volse a retro a rimirar lo passo} \\
    \textit{che non lasciò già mai persona viva.}
\end{center}

\begin{center}
    \textit{Poi ch'èi posato un poco il corpo lasso,} \\
    \textit{ripresi via per la piaggia diserta,} \\
    \textit{sì che 'l piè fermo sempre era 'l più basso.}
\end{center}

La piaggia è un leggero pendio che non è ancora l'effettiva salita.
\\
Il terzo verso, dove un piede è sempre più basso dell'altro,
implica che ci sia ancora una zavorra che lo mantenga vicino al peccato (alla selva).

\begin{center}
    \textit{Ed ecco, quasi al cominciar de l'erta,} \\
    \textit{una lonza leggiera e presta molto,} \\
    \textit{che di pel macolato era coverta;}
\end{center}

Dante incontra la prima delle tre fiere, la lonza.
\sdefinition{Lussuria}{
    La lussuria è un vizio inteso come l'abbandono alle proprie passioni o anche a divertimenti di natura generica, senza il controllo da parte della nostra ragione e della nostra morale.
}
La lonza è leggera (veloce, agile). Essa rappresenta infatti la lussuria.

\begin{center}
    \textit{e non mi si partia dinanzi al volto,} \\
    \textit{anzi 'mpediva tanto il mio cammino,} \\
    \textit{ch'i' fui per ritornar più volte vòlto.}
\end{center}

\begin{center}
    \textit{Temp' era dal principio del mattino,} \\
    \textit{e 'l sol montava 'n sù con quelle stelle} \\
    \textit{ch'eran con lui quando l'amor divino}
\end{center}

Il tempo è il mattino, e la stazione è la primavera.
Secondo la \textit{Genesi} il mondo è stato creato di primavera.

\begin{center}
    \textit{mosse di prima quelle cose belle;} \\
    \textit{sì ch'a bene sperar m'era cagione} \\
    \textit{di quella fiera a la gaetta pelle}
\end{center}

\begin{center}
    \textit{l'ora del tempo e la dolce stagione;} \\
    \textit{ma non sì che paura non mi desse} \\
    \textit{la vista che m'apparve d'un leone.}
\end{center}

Dante ritrova speranza penso che il mattino di primavery sia un momento propizio di inizio.
\\
Dante incontro la seconda delle fiere, il leone.

\begin{center}
    \textit{Questi parea che contra me venisse} \\
    \textit{con la test' alta e con rabbiosa fame,} \\
    \textit{sì che parea che l'aere ne tremesse.}
\end{center}

\sdefinition{Superbia}{
    Radicata convinzione della propria superiorità (reale o presunta) che si traduce in atteggiamenti di orgoglioso distacco o anche di ostentato disprezzo verso gli altr
}
Il leone rappresenta la superbia.

\begin{center}
    \textit{Ed una lupa, che di tutte brame} \\
    \textit{sembiava carca ne la sua magrezza,} \\
    \textit{e molte genti fé già viver grame,}
\end{center}

Immediatamente Date incontro anche la terza fiera, la lupa.
\sdefinition{Avarizia (antica)}{
    La brama di possessi materialistici.
}
La lupa è molto magra. Essa rappresenta l'avariazia, la fame insaziabile e la brama di possessi materiali.

\begin{center}
    \textit{questa mi porse tanto di gravezza} \\
    \textit{con la paura ch'uscia di sua vista,} \\
    \textit{ch'io perdei la speranza de l'altezza.}
\end{center}

Questa gravezza (peso) simboleggia il ritorno verso la sede, perdendo la speranza di salire.

\snote{Alternarsi della speranza}{
    Dante viene attraversato da un continuo alternarsi fra speranza e disperazione.
}

%\begin{center}
%    \textit{E qual è quei che volontieri acquista,} \\
%    \textit{e giugne 'l tempo che perder lo face,} \\
%    \textit{che 'n tutti suoi pensier piange e s'attrista;}
%\end{center}

\pagebreak

\part{Analisi dei testi}

% versi tronchi piani e sfruccioli | accenti tonici etc.
% le rime si considerano tali dalle medesime lettere dopo l'ulima vocale con accento tonico

L'analisi di un testo viene separata nell'\textit{analisi metrica} e nell'\textit{analisi}.

\paragraph{Analisi metrica}

\phantom{ }\vspace{0.1cm}
\sexample{Analisi metric}{
   Sonetto con schema ABBA, ABBA, CDE, CDE.
}


\paragraph{Analisi}

Perifrasi = giro di parole

\end{document}
