\documentclass[a4paper]{article}

\usepackage{amsmath}
\usepackage{amssymb}
\usepackage{parskip}
\usepackage{fullpage}
\usepackage{hyperref}
\usepackage{stellar}

\hypersetup{
    colorlinks=true,
    linkcolor=black,
    urlcolor=blue,
    pdftitle={Italiano},
    pdfpagemode=FullScreen,
}

\title{Italiano}
\author{Paolo Bettelini}
\date{}

\begin{document}

\maketitle
\tableofcontents
\pagebreak

% Autori:
% Dante, Boccaccio, Petrarca
% Machiavelli (sagiistica politica)
% (sagiistica politica) e Ariosto
% Beccaria
% Leopardi
% Leopardi

% Esame scritto: 2 traccie nuove e sceglierne una
% commentare 4 ore di orologio

% Porta un dizionario
% Sufficientemente grande da avere le parole alma guiderdone spirto

% Verifiche:
% 11 ott, Dante 
% 6 dic, Boccaccio(?)



\section{Dante (1265-1321)}

\subsection{Biografia}

La biografia di Dante è molto offuscata e nessuno scritto originale è rimasto.
Questi fattori rendono difficile datare le sue varie opere e contestualizzarle. Inoltre,
è anche difficile validare in maniera precisa le parole esatte scritte dall'autore, siccome i testi che possediamo
sono frutto di trascrizioni.

Dante nasce a Firenze nel 1265 in una piccola nobilità cittadina.
A 12 anni diventa promesso sposo di a Gemma Donati.

A 18 anni incontra Beatrice, dopo averla vista per la prima volta a 9 anni.

Verso l'anno 1295 Dante, si avvicina alla politica.
Si iscrive all'Arte (Arte dei medici e degli speziali), questo è dato dal fatto che essere iscritti ad un Arte
fosse un requisito necessario per esercitare un'attività politica. Dante diventa piore, per cui a capo della cittadina. %%

Prima della nascita di Dante, i Ghibellini sostenevano il potere dell'imperatore, mentre i Guelfi sostenevano quello papa.
Le due parti erano in forte conflitto, e nella battaglia del 1266, muore il figlio dell'imperatore.
I Ghibellini escono quindi di scena quando Dante è appena nato.

Successivamente, i Guelfi si separano in Bianchi e Neri, con un conflitto ancora più forte di quello precedente.

La scena politica fiorentina era dominata dallo scontro fra i Bianchi e i Neri.
Dante, durante il suo priorato, manda in esilio in più violenti dei Neri, fino alla scaduta del suo priorato.
Papa Bonificio VIII manda le truppe di Carlo di Valois, le quali permettono ai Neri di prendere carica al governo.
I bianchi vengono quindi esiliati, fra cui Dante. 

% Boccaccio - primo grande studiso di dante

\subsection{Vita Nova}

\subsubsection{Introduzione}

\sdefinition{Prosimetro}{
    Il \textit{prosimetro} è un testo ibrido, composto da un
    racconto (prosa) intervallato e poesie (versi).
}

La \textit{Vita nova} è il primo prosimetro. Esso racconta la storia d'amore da parte di Dante
nei confronti di Beatrice.
Questa vicenda diventa un modello per questa tipologia di narrativa.

Il significato del titolo indica come Dante consideri l'inizio della sua vita (rinnovata)
quando vide Beatrice per la prima volta.
Il primo contatto amoroso nella poesie è sempre caratterizzato
dall'innamoramento a prima vista.

% Il saluto nel medioevo da parte delle donne
Una volta che la voce dell'interesse di Dante verso Beatrice le giunge, lei gli nega il saluto.
Tuttavia, Dante continua ad esprimere il suo amore verso Beatrice semplicemente
lodandola (scrivendo di lei), completamente senza ricambio di interesse.

Questo libro introduce la simbologia del numero 9 associato a Beatrice.
Questo è dato dal fatto che l'abbia vista per la prima volta a 9 anni, rivista 9 anni dopo,
e altri motivi che vengono descritti. Il numero 9 è anche un simbolo biblico (3 volte la trinità).

\subsubsection{Oltre la spera}

\sdefinition{Sonetto}{
    Il \textit{sonetto} è una poesie composto da 2 quartine e 2 terzine dove tutti i versi sono degli endecasillabi.
}

\section{La Divina Commedia}

\subsection{Introduzione}

\subsection{Inferno - I, II}

\pagebreak

\section{Analisi dei testi}

% versi tronchi piani e sfruccioli | accenti tonici etc.
% le rime si considerano tali dalle medesime lettere dopo l'ulima vocale con accento tonico

L'analisi di un testo viene separata nell'\textit{analisi metrica} e nell'\textit{analisi}.

\paragraph{Analisi metrica}

\phantom{ }\vspace{0.1cm}
\sexample{Analisi metric}{
   Sonetto con schema ABBA, ABBA, CDE, CDE.
}

\paragraph{Analisi}

\end{document}
