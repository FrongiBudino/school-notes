\documentclass[a4paper]{article}

\usepackage{amsmath}
\usepackage{amssymb}
\usepackage{parskip}
\usepackage{fullpage}
\usepackage{hyperref}
\usepackage{stellar}

\hypersetup{
    colorlinks=true,
    linkcolor=black,
    urlcolor=blue,
    pdftitle={Storia},
    pdfpagemode=FullScreen,
}

\title{Storia}
\author{Paolo Bettelini}
\date{}

\begin{document}

\maketitle
\tableofcontents
\pagebreak

% EAN 9788859300434

\section{Storia}

\sdefinition{Storiografia}{
  La \textit{storia} è lo studio, catalogalizzazione e la comprensione degli eventi passati.
}

\section{Periodizzazione}

\sdefinition{Periodizzazione}{
  La \textit{periodizzazione} è l'operazione culturale volta a suddividere la linea temporale in vari intervalli.
}

Le prime periodizzazioni derivano dalle prime religioni monoteiste (Es. nascità di Gesù, calendario islamico).

\section{Fake news storiche}

% Da: Fascismo e fake news

Le fake news sono in genere effimere, ma quelle storiche sono persistenti e
pronfonde nelle persone.

\begin{itemize}
    \item Più una bugia viene ripetuta, più la si può scambiare per verità.
    \item Notizie di oggi viaggiano velocemente, è difficile bloccarle e smentirle.
    \item Comprendere il passato è un modo per comprendere il presente.
    \item Esistono fake news storiche, ancorate ad un argomento preciso.
    \item Bufale storiche vanno contrastate perché falsificano il passato (così come il ricordo e la memoria).
    \item Bufale storiche nascono da osservazioni o testimonianze inesatte, che poi si diffondono in una società pronta ad accoglierle.
    \item Bufale storiche servono ad alimentare emozioni e a rassicurare: credere in un passato positivo può portare la speranza e rischia di creare una prospettiva a cui tendere.
\end{itemize}

Effetti di scardinare le bufale:

\begin{itemize}
    \item Corregere le informazioni sul passato.
    \item Distruggere sicurezze, e ciò può creare incomunicabilità.
    \item Permette di limitare l'ambito di diffusione di queste notizie, che mistificano la memoria e la percezione del presente.
\end{itemize}



\end{document}
